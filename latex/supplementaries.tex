%%% TeX-master: "manuscript"
% !TeX spellcheck=en_GB

\section*{Supplementary Legends}

\subsection*{Supplementary figure S1}

\textbf{The structurally unresolved loop structure of NamPRT.} Sequence alignment of NamPRT of different species cropped to the region around the unresolved loop structure. Coloured rectangles indicate the enzymes present in the species besides NamPRT; blue: NNMT; black: NADA and NNMT; yellow: NADA; green: NamPRT only. Major clades are indicated for better orientation. Number of amino acid indicated at the top refer to the human protein.


\subsection*{Supplementary figure S2}

\textbf{The phylogenetic distribution of NamPRT and NNMT in birds and reptiles is scattered.} The phylogenetic distribution of birds and reptiles was adopted from \cite{Prum2015}. Families are marked with a green circle if they possess NamPRT without NNMT or a blue circle if they possess both NamPRT and NNMT.


\subsection*{Supplementary figure S3}

\textbf{Purification of wt NamPRT and $\Delta$42-51 NamPRT.} A) Elution profile of wildtype NamPRT and mutant $\Delta$42-51 on size-exclusion chromatography using a Superdex 200 16/60 column. B) Coomassie stained denaturating SDS-PAGE analysis of $\Delta$42-51 NamPRT (lane 1) and wt NamPRT (lane 2). 3\,$\mu$g of pooled enzyme eluted from SEC loaded onto the gel. C) The column was calibrated with apronitin 6.5\,kDa, ovalbumine 42.7\,kDa, coalbumine 75\,kDa and blue dextran 2000\,kDa. The partition coefficient (Kav) was determined for each standard (light grey squares) and plotted versus log$_{10}$ molecular weight. The Kav was determined for wt NamPRT and $\Delta$42-51 NamPRT and the apparent molecular weight calculated to be 135\,kDa and 110\,kDa, respectively.


\subsection*{Supplementary figure S4}

\textbf{NMR measurement of NamPRT activity.} NamPRT activity was measured as described in Experimental Procedures and product (NMN) formation was detected using 1D 1H NMR spectroscopy. Inset on the right: molecular structure of NMN with the atom detected by NMR indicated by an arrow. The range used for NMN detection in typical 1D 1H NMR spectra of the enzymatic reactions is shown. NMN quantification was done with the singlet detected at 9.52 ppm. From the top to the bottom, peak detection of NMN standard (200\,µM), wildtype NamPRT (1\,mM Nam and 1\,mM PRPP), mutant $\Delta$42-51 NamPRT (1\,mM Nam and 1\,mM PRPP), wildtype NamPRT with FK866, and mutant $\Delta$42-51 NamPRT with FK866. Incubation with inhibitor FK866 was done for 30\,min at 30\,°C.


\subsection*{Supplementary table S1}

\textbf{Query proteins used for Blast searches.}


\subsection*{Supplementary table S2}

\textbf{Overview of kinetic constants and rate laws used for the construction of the mathematical model.}
