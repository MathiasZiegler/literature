%%% TeX-master: "manuscript"

\section{Discussion}

\subsection{Evolutionary and physiological role of NNMT}

The dynamic interactions among metabolism, signal transduction, and gene regulation is still not very well understood. New approaches are required to disentangle the underlying network, helping us to understand how alterations affect human physiology. We here combine phylogenetic analysis and mathematical modelling of a central metabolic pathway supported by experimental analysis to reveal the likely evolutionary development of the NAD biosynthesis and consumption pathway and the physiological role of the methyltransferase NNMT.

We show that the vitamin B3-degrading enzyme NNMT plays a vital role for the diversification of NAD-dependent signalling reactions and potentially also for NAD homeostasis. Due to the fact that NamPRT has a very high affinity for Nam whereas NNMT has a rather low affinity for it, NNMT is able to remove excess Nam that would lead to inhibition of NAD-dependent signalling reactions while maintaining NAD concentrations and even increasing NAD consumption fluxes.

The affinities measured for the human enzymes seem to be optimal and further increase in NAD consumption rates might only enabled by compartmentalisation of NAD biosynthesis.

In contrast to our results, it was proposed that NNMT is present in yeast (\textit{Saccharomyces cerevisiae} \cite{Anderson2003}. The enzyme was later further characterised as methyltransferase of the transcription elongation factors TEF1 and TEF2 but not Nam \cite{Wlodarski2011}.


\subsection{General applicability}

It is of course impossible to extend our conclusions to other metabolic processes without analysing them in more detail, but we have noted that the metabolism of other vitamins, such as pyridoxal (vitamin B6) metabolism, also contains vitamin degrading enzymes \todo{source}, that might thus have similar roles. Degrading enzymes do in general not receive the same attention as biosynthetic enzymes, reflected by the observation that NNMT has only recently been analysed in more detail, besides the much earlier recognition of NNMT as potential marker for some types of cancer (e.g. \cite{Okamura1998}) and routine clinical measurements of urine methyl-Nam concentrations in the context of different diseases (e.g. \cite{Pumpo2001,Delaney2005}). Our analysis show, however, that degrading enzymes do not solely modify substrates for better excretion, but can play a vital role both in human physiology as well as in the evolutionary development of biological processes.


\subsection{Limitations of our analysis and outlook}

Among others, we neglected the potential effects of the methyl donor S-adenosyl-methionin (SAM) and its precursor methionine in our analysis, although it is most likely contributing yet another regulatory level for Nam availability and thus an additional interaction point between gene regulation and metabolism. It has been shown earlier that in turn, NNMT expression influences protein methylation dependent on methionine availability \cite{Ulanovskaya2013}. One of the challenges in the analysis of this aspect is the availability of in vivo concentration measurements for SAM and the large amount of reactions using it as substrate. The same holds for the analysis of enzyme acetylation versus deacetylation by sirtuins that would enable prediction about enzyme activation or histone state if available.
