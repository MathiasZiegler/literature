%%% TeX-master: "manuscript"
% !TeX spellcheck=en_GB

\section{Discussion}

The dynamic interactions among metabolism, signal transduction, and gene regulation are still not very well understood. New approaches are required to disentangle the underlying network, helping us to understand how alterations affect human physiology. We here combined phylogenetic analysis and mathematical modelling of the NAD biosynthesis and consumption pathway to better understand the evolution of this pathway and especially the role of the methyltransferase NNMT. Our analyses reveal that NNMT plays a vital role for the diversification of NAD-dependent signalling reactions and potentially also for NAD homeostasis. We show that the presence of NNMT enables higher NAD consumption rates in comparison to NamPRT or NADA alone. It furthermore appears that NNMT makes both NAD concentration and NAD consumption relatively independent of other NAD-requiring processes such as cell growth. We furthermore suggest that NNMT exerted an evolutionary pressure on NamPRT, driving the development of the high affinity of NamPRT for its substrates found in the human enzyme \cite{Burgos2008}. These findings shed new light on the role of NNMT, which has earlier been recognized as potential marker for some types of cancer (e.g. \cite{Okamura1998}). Increased NNMT expression in cancer might thus serve to remove Nam produced by increased NAD-dependent signalling. Further analysis is required to better understand which NAD-dependent signalling reactions are affected.

In our analysis, we furthermore noted that NNMT only provides an advantage as long as the NamPRT affinity for Nam is sufficiently high, this implies that treatment with competitive inhibitors that affect the apparent $K_{M}$ would have stronger effects on cells that express NNMT compared to cells that do not express NNMT (see fig.~\ref{fig:NamPRT_affinity_Nam}E and~F). This could be important for cancer treatments as the expression of NNMT is relatively low in normal tissues except liver, whereas several types of cancer overexpress NNMT.

We here neglected the potential effects of the methyl donor S-adenosyl-methionin (SAM) and its precursor methionine. Changing SAM availibility is most likely contributing yet another regulatory level for Nam availability and thus an additional interaction point between gene regulation and metabolism. It has been shown earlier that NNMT expression influences protein methylation dependent on methionine availability \cite{Ulanovskaya2013}. One of the challenges in the analysis of this aspect is the availability of \textit{in vivo} concentration measurements for SAM and the large amount of reactions using it as substrate. A simplified mathematical model of this pathway has been published earlier \cite{Reed2004} providing the basis for future studies.

Beside gaining new insights into the role of NNMT for NAD metabolism, our analyses provided a suggestion why NADA disappears in vertebrates while it co-exists with NamPRT in bacteria. It seems that the combined enzymatic activity of NNMT and NamPRT has been optimised in vertebrates to maintain stable NAD concentration and high NAD consumption flux, making NADA obsolete.


\section{Conclusions}

We here highlight the co-evolution of several enzymes of the NAD pathway with the appearance of NNMT seemingly initiating and driving complex alterations of the pathway such as an increase and diversification of NAD-dependent signalling, followed by an increase in NamPRT substrate affinity (schematic overview see fig.~\ref{fig:evo_events}). This again appears to be accompanied by the loss of NADA in vertebrates and the first gene duplication of NMNATs \cite{Lau2010}. We also noted that the second gene duplication of NMNATs and thus the further compartmentalisation of NAD metabolism is co-occurring with a site-specific positive selection event in NNMT (unpublished results). To our knowledge, this is the first study that comprehensively demonstrates a functional co-evolution of several enzymes of a pathway.
