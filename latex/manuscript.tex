
\RequirePackage[l2tabu, orthodox]{nag}
\documentclass[paper=a4, 12pt]{scrartcl}
\usepackage[utf8]{inputenc}
\usepackage[colorlinks=true,
	linkcolor=black,
	citecolor=black,
	filecolor=black,
	urlcolor=black]{hyperref}
\usepackage{textcomp}
\usepackage[left=1cm,right=5cm,top=2cm,bottom=2cm,nohead,nofoot]{geometry}
\usepackage{todonotes}

\setlength{\marginparwidth}{4cm}

\begin{document}


\noindent
{\huge\sffamily\bfseries NamPRT and NNMT – key drivers of NAD-dependent signalling \par}

\vspace{2cm}

\noindent
Mathias Bockwoldt$^{1}$, Marc Niere$^{2}$, Toni I. Gossmann$^{3}$, Mathias Ziegler$^{2}$ and Ines Heiland$^{1,\S}$

\vspace{1cm}

\noindent
$^{1}$Department of Arctic and Marine Biology, UiT Arctic University of Norway, Naturfagbygget, Dramsveien 201, 9037 Tromsø, Norway

\noindent
$^{2}$Department of Molecular Biology, University of Bergen, Thormøhlensgt. 55
5020 Bergen, Norway

\noindent
$^{3}$Department of Animal and Plant Sciences, Western Bank, University of Sheffield
Sheffield, S10 2TN, United Kingdom

\noindent
§ Corresponding author: ines.heiland@uit.no


\section{Summary}


\section{Keywords}


\section{Highlights}


\section{Introduction}

NAD metabolism represents one of the most critical links that connect cellular signal transduction and energy metabolism. According to Kyoto Encyclopedia of Genes and Genomes KEGG (\url{http://www.genome.jp/kegg/}), more than 20\% of all biochemical reactions use NAD(P), mostly in redox reactions in which NAD(P)+ and NAD(P)H are reversibly interconverted. In addition to redox reactions, various signalling processes consume NAD+ by cleaving the molecule to nicotinamide (Nam) and ADP-ribose \cite{Verdin2015}. NAD+-dependent signalling reactions include poly- and mono-ADP-ribosylation, NAD-dependent protein deacylation by sirtuins as well as the synthesis of calcium mobilizing molecules such as cyclic ADP-ribose \cite{Opitz2015}. These NAD+-dependent signalling processes participate in the regulation of virtually all cellular activities. The enzymes involved in these processes are sensitive to the available NAD+ concentration \cite{Ruggieri2015}, which in turn is dependent on the NAD+/NADH redox ratio. Therefore, NAD+-dependent signalling can act as a transmitter of changes in the cellular energy homeostasis, for example, to regulate gene expression or metabolic activity \cite{Koch-Nolte2009}.

The significance of NAD+-dependent signalling for NAD homeostasis has long been underestimated. However, it has now been established that inhibition of NAD biosynthesis in mammalian cells leads to a rapid decline in NAD concentration suggesting that NAD+-dependent signalling consumes substantial amounts of NAD (Ref?), that is why we later refer to them as NAD-consuming reactions. As it has been shown that NAD concentrations change in a circadian manner \cite{Nakahata2009; Ramsey2009}, the cellular NAD-pool is turned over at least once per day\todo[author=Mathias]{Either the NAD-pool is turned over at least once a day or the concentration is maintained, but both … could be possible but sounds strange.}. To maintain the NAD concentration at physiological levels, NAD biosynthesis needs to proceed at an equally rapid rate. Imbalances in NAD-homeostasis have been linked to various, in particular, ageing-related diseases such as diabetes, neurodegenerative disorders and cancer \cite{Chiarugi2012; Verdin2015}\todo[author=Ines]{NAD turnover}. Various recent studies have demonstrated impressive health benefits of dietary supplementation with intermediates of NAD biosynthesis including NMN and nicotinamide riboside, NR \cite{Belenky2007}\todo[author=Ines]{That is not very recent. Better suggestions?}. Apparently, the exploitation of NAD biosynthetic routes, in addition to the use of nicotinamide as precursor (Fig. 1), results in increased NAD concentrations that stimulate NAD+-dependent signalling processes, in particular, protein deacetylation by sirtuins. \todo[author=Ines]{Shorten introduction}

In mammals, NAD biosynthesis predominantly relies on nicotinamide \cite{Bogan2008}\todo[author=Ines]{Not the perfect reference maybe}, a form of vitamin B3, as precursor. In the first step, Nam is converted to the mononucleotide, NMN, by Nam phosphoribosyltransferase (NamPRT) using phosphoribosylpyrophosphate (PRPP) as co-substrate. The key role of this enzyme is related to the nature of NAD+-dependent signalling reactions, namely, that in all these reactions Nam is cleaved off and potentially lost. If the released Nam were not recycled into NAD biosynthesis, humans would require a much higher daily vitamin B3 intake than the 15 mg that are the current daily recommendation (ref) \todo[author=Ines]{I have not been able to find a reference for that so far.}. The nearly complete recycling of Nam is achieved by an extraordinary high affinity of NamPRT to Nam, the Km being in the low nanomolar range \cite{Burgos2008}. Despite the importance of its salvage, Nam can also be marked for excretion by methylation \todo[author=Mathias]{It’s not too surprising. If there was no way of removing parts, we would accumulate more and more, as we still take up 15 mg every day.}. Indeed, the presence of nicotinamide N-methyltransferase (NNMT) in vertebrates \cite{Gossmann2012} is among the most enigmatic and counterintuitive features of NAD metabolism. Why is there one enzyme (NamPRT) seemingly optimized to recycle even the faintest amounts of Nam back into NAD synthesis, while at the same time there is another one (NNMT) that seems to have no metabolic function other than to remove Nam from NAD metabolism\todo[author=Ines]{This paragraphs to the start including a short statement about the purpose of the analysis.}? This puzzle becomes even more intriguing when considering that the majority of lower organisms and plants deaminated Nam to nicotinic acid (NA) before it can enter NAD biosynthesis via the Preiss-Handler pathway (Fig. 1). Previous phylogenetic analysis show that the two Nam recycling pathways are both ancestral showing a scattered distribution in bacteria \cite{Gazzaniga2009}. In contrast, NNMT has so far only been found in animals \cite{Gossmann2012}. The analysis of eukaryotic pathway evolution was previously limited by the low number of available genome sequences. A comprehensive evolutionary analysis is, however, needed to fully understand the phylogeny of NAD metabolism, in particular, Nam metabolizing enzymes and their relationship to signalling pathways.

This analysis may also help to unravel, how biosynthesis and signalling are coupled and influence each other, as this is not well understood so far. It is widely assumed that a major driving force of NAD+-consuming signalling reactions is the cellular NAD concentration. Most enzymes using NAD in signalling reactions that have been characterized in detail and it was shown that most sirtuins and PARPs are subject to inhibition by nicotinamide \cite{Borra2004; Ko2012}. To promote NAD+-dependent signalling reactions, the balance between elevated NAD levels and accumulating nicotinamide concentrations needs to be kept such that the inhibitory effect of Nam does not override the kinetic stimulation by high NAD levels \todo[author=Mathias]{Well… To promote [signalling] reactions, Nam simply has to be turned into NAD as quickly as possible. Besides that it sounds, form the description, as a self-regulating process.}. Given the complex nature of the NAD metabolome, it is not trivial to predict the optimal conditions for efficient NAD dependent signalling. However, it is obvious that Nam-converting enzymes must play a key role in this regard.

In the present study, we have conducted a comprehensive phylogenetic analysis of the genes that encode Nam-converting enzymes. The most surprising result was that with the emergence of Deuterostomes, there has been a strong tendency towards the co-existence of NamPRT and NNMT. Moreover, this selection for the co-existence of NamPRT and NNMT was accompanied by a marked increase in the number of genes encoding NAD+-consuming signalling reactions. To explain this observation, we built a kinetic model of NAD metabolism based on available kinetic parameters. This mathematical model provided a powerful tool to analyse the relationship between NAD biosynthetic and signalling fluxes when different subsets of Nam-converting enzymes are present. The model demonstrated that NNMT has a critical role to maintain high NAD+-consuming signalling fluxes by preventing accumulation of inhibitory Nam. At lower Nam concentrations, the impact of NNMT is minimal, because of the very high affinity of NamPRT. Taken together, our analyses suggest that the co-existence of NamPRT and NNMT has been a prerequisite to enable the evolutionary development of versatile NAD+-dependent signalling mechanisms present in vertebrates.


\section{Results}

\subsection{Phylogenetic analysis of NAD biosynthesis and consumption}

As shown in Fig. 1, NAD can be synthesized using several routes from three main precursors: tryptophan, nicotinamide (Nam) and nicotinic acid (NA). Nam and NA are together known as vitamin B3 or niacin. Alternatively, nicotinamide ribose (NR) can be used omitting the energetically unfavourable reaction of Nam phosphoribosyltransferase (NamPRT), requiring nicotinamide ribose kinase (NRK) instead \cite{Bogan2008}. As in humans only 1\% of the tryptophan taken up with our diet is converted into NAD, vitamin B3 and to a lower extend NR are essential components of our diet, with Nam being the major NAD precursor in humans. Nam is also the product of NAD consuming signalling reactions such as sirtuins (NAD-dependent histone deacetylases) or PARPs (poly ADP ribosylases).

For the recycling of Nam, two different pathways exist. The pathway found in yeast and plants is using a four-step pathway starting with the deamination of Nam to nicotinic acid by Nam deamidase (NADA). The other three enzymes comprise the Preiss-Handler pathway that also exists in vertebrates. The recycling pathway found in mammals directly converts Nam into the corresponding mononucleotide (NMN) a reaction catalysed by NamPRT and driven by a non-stoichiometric ATP-hydrolysis. A similar reaction catalysed by an evolutionary related enzyme NAPRT, converts NA into the NA mononucleotide in the Preiss-Handler pathway. NMN and NAMN are converted into dinucleotides by the Nam/NA adenylytransferases (NMNATs). The recycling pathway via NA finally requires an amination step catalysed by NADsynthase, driven by the conversion of ATP to AMP producing pyrophosphate. Even though the latter pathway seems to be very inefficient, it is the pathway preferentially used by most bacteria, fungi and plants (see Figure 2A), whereas most metazoans recycle Nam using NamPRT.

Analysing the phylogeny of the NAD recycling enzymes in Metazoa in more detail reveals that not only does NamPRT replace NADA, but in most organisms, especially in Deuterostomia, NamPRT is found together with the Nam methyltransferase NNMT (Figure 2B). NNMT methylates Nam to methyl-Nam that is in mammals excreted with the urine, thus removing Nam from recycling. NNMT seems to have arisen de novo in the common ancestor of Ecdysozoa and Lophotrochozoa, as we could not find any gene with considerable similarity in fungi or plants. Nematodes are the only organisms where we find NNMT together with NADA without NamPRT being present. In Deuterostomia the only large class that does only have NamPRT and seems to have lost NNMT again are Sauropsida and here especially birds. The reason why a lot of birds do not encode NNMT remains unclear, as the appearance is quite scattered (not shown)\todo[author=Mathias]{Maybe as supplementary figure?}. It might be related to the excretion system, as the product of NNMT methyl-nicotinamide is in mammals excreted with the urine. There are some species where we could not find NamPRT or NADA but NNMT, we assume that this is due to incomplete genomes in the database\todo[author=Ines]{Not sure we should discuss Branchisotoma and Tunicata here}.

In addition to the phylogenetic distribution of the two Nam salvage enzymes NADA and NamPRT, we looked at phylogenetic diversity of enzymes catalysing NAD-dependent signalling reactions. To do so we used the previously established classification into 10 different families \cite{Gossmann2012} (For details see materials and methods and supplementary information). The numbers in Figure 2B denote the average number of NAD-dependent signalling enzyme families we found in each taxa. With the exception of Cnidaria and Lophotrochozoa, we find in Protostomia on average 3 to 4 families, whereas most Deuterostomia have on average more than 8 families with an increasing diversification of enzymes within at least some of these families \cite{Gossmann2014}.

Taken together, we found that the presence of NamPRT and NNMT coincides with an increased diversification of NAD-dependent signalling. That is surprising as intuitively, one would expect that a decrease in precursor concentration caused by the precursor removal through NNMT, should cause a decrease of NAD availability and consequently less active NAD dependent signalling.


\subsection{Dynamics of NAD biosynthesis and consumption}

So why does the diversity of NAD-dependent signalling increase? And why does NADA disappear in Deuterostomia although it is the predominant pathway in bacteria, plants and fungi? Given the complexity of the NAD metabolic network, this question is difficult to be comprehensively addressed experimentally. Thus, to answer these questions we built a dynamic model of NAD metabolism using existing kinetic data from the literature (details see materials and methods and supplementary material).

To be able to compare metabolic features of evolutionary quite different systems in our simulations, and as we have limited information about expression levels of enzymes or changes of kinetic constants during evolution, we initially used the kinetic constants found for yeast or human enzymes for all systems analysed and used equal amounts of enzymes for all reactions. Wherever possible we did not only include substrate affinities but also known product inhibition or inhibition by downstream metabolites. As we assume that cell growth is, besides NAD-consuming reactions, a major driving force for NAD biosynthesis, we analysed different growth rates (cell division rates) by simulating different dilution rates for all metabolites. We assume furthermore that Nam availability is different for different organisms and thus in addition analysed this simulating different Nam import rates.

The pathway using NADA and recycling Nam via NA is superior \todo[author=Mathias]{What is meant with superior? NAD consumption flux is higher or lower with NNMT, depending on the condition. Free NAD concentration is higher without NNMT, but is that superior?} both in terms of steady state NAD concentration and NAD consumption flux (representing the activity of NAD-dependent signalling) in the absence of NNMT (Figure 3). In the presence of NNMT, the picture is slightly different (Figure 3C-D), even though NADA is still superior to NamPRT under most conditions \todo[author=Ines]{Check if not under all shown}\todo[author=Mathias]{Reply: From fig 4C, it looks like the flux is decreased at very high cell division rates while the Nam import rate is low.}. Only if Nam availability is very low, NamPRT is performing better because of its high substrate affinity. However, the disadvantage of NamPRT at equal amounts of enzyme, can be compensated by an increased expression of NamPRT. Although, to reach similar NAD concentrations in our model, NamPRT expression has to be tenfold higher than the expression of NAPRT (Figure S1). The NADA expression required is very low due to its high turnover. This might provide an explanation why we find NADA and with that the pathway via NA predominantly in bacteria, yeast, and plants, organisms that show high cell division rates. Under these conditions, protein expression costs are assumed to have a high impact on metabolic performance \todo[author=Mathias]{source} and thus pathways where low enzyme expression suffices, might be favourable, even though the pathway via NA is energetically less efficient.

NADA seems to be able to maintain higher NAD concentrations than NamPRT. This figure changes if we simulate two organisms or cells that are in direct competition for Nam. Under these conditions, the organism expressing NamPRT has a competitive advantage above NADA-expressing organisms, but this only holds if the competing organisms expresses NamPRT together with NNMT. (new Figure) This observation might explain, why NADA disappeared in Metazoa together with the appearance of NNMT. It might also explain, why many bacteria associated with mammals harbour NamPRT instead of NADA\todo[author=Mathias]{Needs some figure or source}.

Although these simulations already provide an idea of why we find NamPRT predominantly in combination with NNMT, we still do not know why this development coincides with an increased diversification of NAD-consuming enzymes. When we simulate the presence or absence of NNMT in the presence of NamPRT, we see that the impact of NNMT on NAD concentration is relatively small (Figure 4D), but we see that that NNMT increases the NAD consumption flux under most conditions in the presence of NamPRT (Figure 4C). NAD consumption flux can be further increased by increasing the expression NamPRT, which also compensates the slight reduction in NAD concentration in the presence of NNMT (see Figure S2).

These findings can be explained when looking in more detail into the kinetic parameters of NamPRT and NAD-consuming enzymes such as Sirt1p. The ability to maintain high a NAD concentration in the presence of NNMT and at low Nam availability, is due to the very high affinity of NamPRT for its substrate, having a half saturation constant (Km) in the low nanomolar range. The increase of NAD consumption flux is caused by the fact that most NAD-consuming enzymes are inhibited by their product Nam, which is also the reason why the presence of NNMT enables higher NAD consumption fluxes.

As the substrate affinity of NamPRT for Nam is extremely high (with a Km in the low nanomolar range) and as this might not have been the case throughout evolution, we analysed the effect of the NamPRT Km on NAD steady state concentration and NAD consumption flux, leaving all other kinetic parameters constant. In the absence of NNMT (Figure 5A-B) the Km has very little effect on steady state NAD concentration and NAD consumption flux. Without NNMT, NAD concentration and consumption flux are both considerably affected by cell division rates, at least if the enzyme expression is kept constant. This is of course an artificial scenario, as one would assume organisms to regulate enzyme expression to achieve similar levels of metabolite concentrations instead. In the absence of NNMT there appears furthermore to be a trade-off between achievable steady state NAD concentration and NAD consumption flux.

In the presence of NNMT, NAD consumption flux and NAD concentration increases with decreasing Km values (Figure 5C-D). And we note, that both the NAD steady state concentration and the consumption flux are relatively stable over a wide range of cell division rates even at constant expression levels of the involved enzymes, suggesting that NNMT might have an important role to maintain homeostasis of NAD metabolism.

When we compare NAD consumption and NAD concentration with and without NNMT with two different substrate affinities of NamPRT, we see that at a low affinity (Km of 100 nM, which is in the range of the Km of NAPRT for its substrate, or the Km of NADA for Nam), NAD consumption flux is only higher with NNMT at low cell division rates, whereas at high division rates, higher NAD consumption fluxes are achieved without NNMT (Figure 5 E-F). This might explain why we do not find NNMT in organism that tend to have high growth rates\todo[author=Ines]{I am not sure if we should go even further suggesting that decreasing substrate affinity by using a competitive inhibitor such as FK866 will influence the consumption rate in fast growing cells much more than the one in slow growing cells, being potentially relevant for cancer therapy. – Discussion}. We furthermore find, that the competitive advantage of NamPRT of NADA is only present at sufficiently high affinity of NamPRT.

Our analysis also suggests that NNMT might have exerted an evolutionary driving force on the substrate affinity of NamPRT, explaining the extreme values found for the human enzyme\todo[author=Ines]{Move to discussion?}.

The pathway dynamics are of course not solely dependent on one enzyme. Thus, what is the impact of the substrate affinity of NNMT that is competing with NamPRT for the same substrate? In Figure 6A we see that the substrate affinity values found in the human enzymes (indicated by black asterisks) are actually optimal with respect to both achievable steady state NAD concentration and consumption fluxes. Thus, a further increase of the affinity of NamPRT for Nam would not provide any advantage.


\subsection{Sequence variance acquired in metazoans enhances substrate affinity}

To see whether we can find sequence variations in the protein sequence of NamPRT that indicate evolutionary changes of NamPRT upon the appearance of NNMT, we created a multiple sequence alignment of selected eukaryotic NamPRT sequences. As shown in Figure 7A and Supplementary Figure 3\todo[author=Ines]{To be created}, Deuterostomia that have only NamPRT and NNMT (indicated by the number 6 in parenthesis \todo[author=Ines]{Maybe rather indicate this by colour and the kingdom through abbrev?}) have an insert of ten amino acids corresponding to positions 43 to 52 of the human enzyme. Looking at the crystal structure of human NamPRT this sequence insertion corresponds to a region that has not been structurally resolved in any of the currently available crystal structures (e.g. \cite{Wang2006} structure visualisation Figure 7B) and its function is unclear. The unresolved loop structure overlaps with a predicted weak nuclear localisation signal (figure 7A), that is not present without the insertion. The loop is in addition connected to one of the beta-sheets involved in substrate binding, potentially affecting the affinity or turnover of the enzyme.

The observed evolutionary change in the primary sequence of NamPRT could therefore have had different effects, that we wanted to test experimentally. We first investigated whether the deletion of the amino acids 43 to 52 has an effect on the localisation of the human enzyme, expressing a …. mutant lacking this sequence in HeLa cells. We could, however, not detect any changes in subcellular localisation (Figure 7C) compared to the wildtype fusion construct. We therefore conclude that the partial nuclear localisation of NamPRT is not affected by the deletion. We then tested the enzymatic activity of the NamPRT mutant recombinantly expressed in E. coli. As shown in Figure 7D the enzyme still forms a dimer, thus seems to be folded correctly. The enzymatic activity is, however, much lower compared to the wildtype enzyme. Using higher concentrations of Nam, it appears that the mutant is not saturated at 100 \textmu M, pointing to a decreased substrate affinity of the mutant enzyme.

\subsection{Comments and parts of old version not included yet}

Chordata: NADA disappears. These analyses also support the view that Tunicata and Branchiostoma are not part of the phylum chordata\todo[author=Mathias]{Is this view in any way argued upon by anyone??}.

In the evolutionary context, an additional question arises: Why do only a few organisms, mostly less complex animals, possess the gene for NADA in addition to NamPRT and NNMT. When including NADA into the simulations, we see that the effect of NADA on NAD consumption is very limited in the presence of NNMT (old Figure 4F) even at high expression levels (see old Suppl. Figure S1C and D) and could be compensated by increased expression of NamPRT (not shown).

Until this point we have neglected compartmentalisation of the pathway. We do however know from previous studies that in early vertebrate development a compartmentalisation of the pathway has occurred reflected by a gene triplication of NMNAT and the occurrence of compartment-specific domains called ISTIDs (Lau, 2010). Looking at the evolutionary timepoint \todo[author=Toni]{This time point can only be reconstructed, but is unknown. I think a figure with the likely reconstruction would help to follow the argumentation} of appearance of NNMT and the gene triplication of NMNATs, we see that NNMT occurs prior to the gene triplication and the first occurrence of ISTIDs and we have confirmed this in our own analysis (not shown).


\section{Discussion}

\subsection{Evolutionary and physiological role of NNMT}

The dynamic interactions among metabolism, signal transduction, and gene regulation is still not very well understood. New approaches are required to disentangle the underlying network, helping us to understand how alterations affect human physiology. We here combine phylogenetic analysis and mathematical modelling of a central metabolic pathway supported by experimental analysis to reveal the likely evolutionary development of the NAD biosynthesis and consumption pathway and the physiological role of the methyltransferase NNMT.

We show that the vitamin B3-degrading enzyme NNMT plays a vital role for the diversification of NAD-dependent signalling reactions and potentially also for NAD homeostasis. Due to the fact that NamPRT has a very high affinity for Nam whereas NNMT has a rather low affinity for it, NNMT is able to remove excess Nam that would lead to inhibition of NAD-dependent signalling reactions while maintaining NAD concentrations and even increasing NAD consumption fluxes.

The affinities measured for the human enzymes seem to be optimal and further increase in NAD consumption rates might only enabled by compartmentalisation of NAD biosynthesis.


\subsection{General applicability}

It is of course impossible to extend our conclusions to other metabolic processes without analysing them in more detail, but we have noted that the metabolism of other vitamins, such as pyridoxal (vitamin B6) metabolism, also contains vitamin degrading enzymes \todo[author=Mathias]{source}, that might thus have similar roles. Degrading enzymes do in general not receive the same attention as biosynthetic enzymes, reflected by the observation that NNMT has only recently been analysed in more detail, besides the much earlier recognition of NNMT as potential marker for some types of cancer (e.g. Ref?) and routine clinical measurements of urine methyl-Nam concentrations in the context of different diseases (e.g. Ref?). Our analysis show, however, that degrading enzymes do not solely modify substrates for better excretion, but can play a vital role both in human physiology as well as in the evolutionary development of biological processes.


\subsection{Limitations of our analysis and outlook}

Among others, we neglected the potential effects of the methyl donor S-adenosyl-methionin (SAM) and its precursor methionine in our analysis, although it is most likely contributing yet another regulatory level for Nam availability and thus an additional interaction point between gene regulation and metabolism. It has been shown earlier that in turn, NNMT expression influences protein methylation dependent on methionine availability \cite{Ulanovskaya2013}. One of the challenges in the analysis of this aspect is the availability of in vivo concentration measurements for SAM and the large amount of reactions using it as substrate. The same holds for the analysis of enzyme acetylation versus deacetylation by sirtuins that would enable prediction about enzyme activation or histone state if available.


\section{Experimental Procedure}

\subsection{Dynamic modelling}

Kinetic parameters (substrate affinity (Km) and turnover rates (kcat), substrate and product inhibitions) were retrieved from the enzyme database BRENDA and additionally evaluated by checking the original literature especially with respect to measurement conditions. Parameter values from mammals were used if available. For enzymes not present in mammals, values from yeast were used. The full list of kinetic parameters including reference to original literature can be found in Supplementary table 1. For NMNAT, the previously developed rate law for substrate competition was used \cite{Schauble2013}. Despite these modifications, Henri-Michaelis-Menten kinetics were used for all reactions except the import and efflux of Nam, which were simulated using constant flux and mass action kinetics, respectively. All simulations were performed using the steady state calculation and parameter scan options provided by COPASI 4.22 \cite{Hoops2006}. The model will be available at the Biomodels database accession number \todo[inline]{xxx}. Related figures were generated using gnuplot\todo[author=Mathias]{Version?}.


\subsection{Phylogenetic Analysis}

NADA, NamPRT, and NNMT enzymes or enzyme candidates were identified using Blastp with known enzymes against the NCBI non-redundant protein sequence database (nr). A list of functionally verified enzymes used as templates is given in supplementary table 2. This table also includes the length cut-offs for identified enzymes. The e-value cut-off was 1e-30 for all enzymes. Blastp parameters were set to yield maximum 20,000 target sequences, using the BLOSUM62 matrix with a word size of 6 and gap opening and extension costs of 11 and 1, respectively. Low-complexity filtering was disabled. Obvious sequence contaminations were removed by manual inspection of the results. The taxonomy IDs of the species for each enzyme was derived from the accession2taxonomy database provided by NCBI. Scripts for creating, analysing, and visualising the phylogenetic tree were written in Python, using the ETE3 toolkit (Huerta-Cepas, 2010).


\subsection{Generation of eukaryotic expression vectors encoding C-terminally FLAG-tagged NamPT proteins}

The open reading frame (ORF) encoding human NamPRT was inserted into pFLAG-CMV-5a (Merck - Sigma Aldrich) via EcoRI/BamHI sites. Using a PCR-based approach, this vector provided the basis for the generation of a plasmid encoding a NamPRT deletion mutant lacking amino acid residues 42-51 (NamPRT-$\Delta$42-51). The sequences of the inserted ORFs were verified by DNA sequence analysis.


\subsection{Transient transfection, immunocytochemstry and confocal laser scanning microscopy}

HeLa S3 cells cultivated in Ham’s F12 medium supplemented with 10\% (v/v) FCS, 2 mM L-glutamine, and penicillin/streptomycin, were seeded on cover slips in a 24 well plate. After one day, cells were transfected using Effectene transfection reagent (Qiagen) according to the manufacturer’s recommendations. Cells were fixed with 4\% paraformaldehyde in PBS 24 hours post transfection, permeabilised (0.5\% (v/v) Triton X-100 in PBS) and blocked for one hour with complete culture medium. After overnight incubation with primary FLAG-antibody (mouse M2, Sigma-Aldrich) diluted 1:2500 in complete medium, cells were washed and incubated for one hour with secondary AlexaFluor 594-conjugated goat anti mouse antibody (ThermoFisher, Invitrogen) diluted 1:1000 in complete culture medium. Nuclei were stained with DAPI and the cells washed. The cover slips were mounted on microscope slides using ProLong Gold (ThermoFisher, Invitrogen). Confocal laser scan imaging of cells was performed using a Leica TCS SP8 STED 3x microscope equipped with a 100x oil immersion objective (numerical aperture 1.4).


\section{Figure Legends}

\subsection{Figure1}

Schematic overview of NAD biosynthesis and consumption. NAD can be synthesized using several routes from altogether three main precursors: tryptophan (Trp), nicotinamide (Nam), and nicotinic acid (NA). Nam and NA are together known as vitamin B3 or niacin. To a lesser extend, nicotinamide ribose (NR) can be used omitting the energetically unfavourable reaction of Nam phosphoribosyltransferase (NamPRT), requiring nicotinamide ribose kinase (NRK) instead. Only 1\% of the tryptophan taken up with our diet is converted into NAD, thus, vitamin B3 and, to a lower extend, NR are essential components of our diet, with Nam being the major NAD precursor in humans\todo[author=Mathias]{Whole sentence not necessarily part of the legend}. Nam is furthermore the product of NAD-consuming signalling reactions such as sirtuins (NAD-dependent histone deacetylases) or PARPs (poly-ADP-ribosylases).

For the recycling of Nam, two different pathways exist. The pathway found in yeast and many bacteria is using a four-step pathway starting with the deamination of Nam to nicotinic acid (NA) by Nam deamidase (NADA). The other three enzymes comprise the Preiss-Handler pathway that also exists in vertebrates. The recycling pathway found in vertebrates directly converts Nam into the corresponding mononucleotide (NMN) a reaction catalysed by NamPRT and driven by a non-stoichiometric ATP-hydrolysis. A similar reaction catalysed by an evolutionary related enzyme NAPRT, converts NA into the NA mononucleotide (NAMN). NMN and NAMN are converted into dinucleotides by the Nam/NA adenylytransferases (NMNATs). The recycling pathway via NA finally requires an amination step catalysed by NADsynthase (NADS), driven by the conversion of ATP to AMP producing pyrophosphate.


\subsection{Figure 2}

Evolutionary distribution of NADA, NNMT and NamPRT and their relation to the number of NAD consumers. A) Distribution of NADA, NNMT and NamPRT in selected major taxa. NADA is dominant in Bacteria, Fungi, and Plants (Viridiplantae), whereas NamPRT together with NNMT is dominant in Deuterostomia. Numbers at the pie charts show, how many species of the taxon possess the respective enzyme combination indicated by the colour explained in the lower right of the figure. Below the taxon name, the number of species in that taxon is given.

B) Common tree of selected taxa within the Metazoa, including 334 species. The pie charts indicate the distribution of species within the respective taxon that have the enzyme combination indicated by the colour, explained in the lower right. The size of the pie charts is proportional to the logarithm of the number of species analysed in the particular taxon. The numbers below the taxon names indicate the average number of NAD-consuming enzyme families found in all sub-taxa. The branch length is arbitrary.


\subsection{Figure 3}

NAD consumption flux and NAD concentration in simulations of organisms with NADA(?) and with and without NNMT at different Nam import and cell division rates.


\subsection{Figure 4}

Impact of NNMT on NAD consumption and NAD concentration in simulations of organisms with NADA or NamPRT.


\subsection{Figure 5}

NAD consumption flux and NAD concentration in simulations of organisms with NamPRT and with and without NNMT at different KMs for NamPRT and cell division rates.


\subsection{Figure 6}

The substrate affinity of NNMT and NamPRT have opposite effects on NAD consumption (A) and concentration (B), as would be expected. The affinities previously measured for human enzymes (indicated by a black asterisk) appear to be close to optimal, as further improvements would have little or no effect on NAD consumption or concentration.


\subsection{Figure 7}

Deuterostomia that encode NNMT show a sequence insertion in the N-terminal region of NamPRT. A) Multiple sequence alignment of NamPRT of selected deuterostomes show a sequence insertion in organisms that encode NamPRT and NNMT. (Species names are given and the number in parenthesis indicates the enzyme combination encoded in these species; 3 – NADA and NamPRT, 6 – NamPRT and NNMT, 7 – NADA, NamPRT and NNMT. \todo[author=Mathias]{Note to myself: replace the numbers by something better and rewrite this part.}) The nuclear localisation signal (NLS\todo[author=Mathias]{Note to myself: Write “predicted NLS” in figure}) indicated, was predicted using using cNLS-mapper (\url{http://nls-mapper.iab.keio.ac.jp/cgi-bin/NLS_Mapper_form.cgi} \cite{Kosugi2009}).

B) The inserted region is not resolved in crystal structures of human NamPRT and thus appears to be a flexible loop structure at the surface of the NamPRT dimer, coloured in red. The visualisation is based on a structure prediction of SWISS-MODEL \cite{Arnold2006; Biasini2014} of the sequence of the human NamPRT (P43490) using the model 2H3D as template \cite{Wang2006}.

C) The localisation of wildtype NamPRT varies between pure cytosolic and mixed cytosolic-nuclear localisation. This is not changed in the mutant missing the insert that corresponds to the unresolved loop structure.


\end{document}
