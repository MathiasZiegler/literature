
\RequirePackage[l2tabu, orthodox]{nag}
\documentclass[paper=a4, 12pt]{scrartcl}
\usepackage[utf8]{inputenc}
\usepackage[colorlinks=true,
	linkcolor=black,
	citecolor=black,
	filecolor=black,
	urlcolor=black]{hyperref}
\usepackage{textcomp}
\usepackage[left=1cm,right=5cm,top=2cm,bottom=2cm,nohead,nofoot]{geometry}
\usepackage{todonotes}

\setlength{\marginparwidth}{4cm}

\begin{document}


\noindent
{\huge\sffamily\bfseries NamPRT and NNMT – key drivers of NAD-dependent signalling \par}

\vspace{2cm}

\noindent
Mathias Bockwoldt$^{1}$, Dorothee Houry$^{2}$,  Marc Niere$^{2}$, Toni I. Gossmann$^{3}$, Mathias Ziegler$^{2}$ and Ines Heiland$^{1,\S}$

\vspace{1cm}

\noindent
$^{1}$Department of Arctic and Marine Biology, UiT Arctic University of Norway, Naturfagbygget, Dramsveien 201, 9037 Tromsø, Norway

\noindent
$^{2}$Department of Molecular Biology, University of Bergen, Thormøhlensgt. 55
5020 Bergen, Norway

\noindent
$^{3}$Department of Animal and Plant Sciences, Western Bank, University of Sheffield
Sheffield, S10 2TN, United Kingdom

\noindent
§ Corresponding author: ines.heiland@uit.no


\section{Summary}

NAD is best known as cofactor in redox reactions, but it is also substrate of NAD-dependent signalling reactions that consume NAD and release nicotinamide (Nam). In eukaryotes, two different Nam salvage pathways exist. While in lower organisms the initial deamidation of Nam is prevalent, in animals the direct conversion of Nam to the mononucleotide by Nam phosphoribosyltransferase (NamPRT) dominates and eventually remains as the single Nam recycling route in vertebrates.
Strikingly, loss of the deamidation pathway in early vertebrates is accompanied by the occurrence of a new enzyme that marks Nam for excretion by methylation – nicotinamide N-methyltransferase (NNMT). The physiological relevance of NNMT is still enigmatic. Why is there an enzyme that removes Nam from recycling, seemingly not having any other physiological role? And why is the occurrence of NNMT accompanied by a diversification of NAD-consuming enzymes?
We have used mathematical modelling approaches to resolve these counterintuitive observations. Our results indicate that NNMT is required to enable high NAD consumption fluxes necessitated by the increasing diversification of NAD-dependent signaling pathways. This kinetic regulation requires a high substrate affinity of the key enzyme for Nam salvage, NamPRT. Indeed, the affinity of NamPRT to Nam has previously been measured to be in the nanomolar range. Mathematical modelling supports the hypothesis that NNMT exerted an evolutionary pressure on NamPRT enforcing the development of its unusually high substrate affinity. Using multiple sequence alignments, we identified a sequence insertion, first occurring in vertebrates, that parallels an - experimentally verified - increase in the substrate affinity of the enzyme. Further simulations show that the deamidation pathway became obsolete owing to the high substrate affinity of NamPRT. Collectively, our results illustrate a close evolutionary relationship between NAD biosynthesis and the diversification NAD-dependent signaling pathways, potentially driven by the concomitant occurrence of a regulator of Nam salvage, NNMT.

\section{Keywords}

%%% TeX-master: "manuscript"
% !TeX spellcheck=en_GB

\section{Introduction}

NAD metabolism represents one of the most critical links that connect cellular signal transduction and energy metabolism. Even though best known as cofactor for various redox-reactions , NAD is involved in a number of signalling processes that consume NAD$^{+}$ by cleaving the molecule to nicotinamide (Nam) and ADP-ribose \cite{Verdin2015}. These NAD-dependent signalling reactions include but are not limited to poly- and mono-ADP-ribosylation, NAD-dependent protein deacylation by sirtuins as well as the synthesis of calcium-mobilizing molecules such as cyclic ADP-ribose \cite{?}. These NAD-dependent signalling processes participate in the regulation of virtually all cellular activities. The enzymes involved in these processes are sensitive to the available NAD concentration \cite{Ruggieri2015}, which in turn is dependent on the NAD$^{+}$/NADH redox ratio. Therefore, NAD-dependent signalling can act as a transmitter of changes in the cellular energy homeostasis, for example, to regulate gene expression or metabolic activity \cite{Koch-Nolte2009}.\todo[author=Mathias B.]{Please add more references to the first paragraph. I think there are better references then my own review.}

The significance of NAD-dependent signalling for NAD homeostasis has long been underestimated. However, it has now been established that substances affecting NAD biosynthesis lead to a rapid decline of the NAD concentration \cite{Buonvicino2018} suggesting that NAD-dependent signalling consumes substantial amounts of NAD, which is why we later refer to them also as NAD-consuming reactions. The resulting NAD turnover differs in a cell type specific manner and can lead to an NAD half-life as short as 2 hours \cite{Liu2018}. To maintain the NAD concentration at physiological levels, NAD biosynthesis needs to act at an equally rapid rate. Imbalances in NAD homeostasis have been linked to various, in particular, ageing-related diseases such as diabetes, neurodegenerative disorders and cancer \cite{Chiarugi2012,Verdin2015}. Several recent studies have demonstrated impressive health benefits of dietary supplementation with intermediates of NAD biosynthesis including NMN and nicotinamide riboside, (NR) \cite{Yoshino2018}. Apparently, the exploitation of NAD biosynthetic routes, in addition to the use of nicotinamide as precursor (Fig. 1), results in increased NAD concentrations that stimulate NAD-dependent signalling processes, in particular, protein deacetylation by sirtuins \cite{North2004}.

Due to the constant release of Nam through NAD-consuming signalling reactions, the NAD salvage pathway using Nam as precursor is the most important NAD synthesis pathway. If Nam would not be constantly recycled into NAD, humans would require a much higher daily vitamin B3 intake than the 16 mg that are the current daily recommendation \cite{CommissionofEuropeanCommunities2008}. In the first step of the salvage pathway, Nam is converted to the mononucleotide, NMN, by Nam phosphoribosyltransferase (NamPRT) using phosphoribosyl pyrophosphate (PRPP) as co-substrate. The nearly complete recycling of Nam is achieved by an extraordinary high affinity of NamPRT to Nam, the $K_{M}$ being in the low nanomolar range \cite{Burgos2008}. Despite the importance of its salvage, Nam can also be marked for excretion by methylation. The presence of nicotinamide N-methyltransferase (NNMT) in vertebrates \cite{Gossmann2012FEBS} is among the most enigmatic and counterintuitive features of NAD metabolism. Why is there one enzyme (NamPRT) seemingly optimised to recycle even the faintest amounts of Nam back into NAD synthesis, while at the same time there is another enzyme (NNMT) that seems to have no metabolic function other than to remove Nam from NAD metabolism, This puzzle becomes even more intriguing when considering that the majority of lower organisms and plants deamidated Nam to nicotinic acid (NA) using the nicotinamide deamidase (NADA) before it can enter NAD biosynthesis via the Preiss-Handler pathway (Fig. 1).

We here present a phylogenetic analysis of NAD pathway in eukaryotes that. In contrast to our previous analysis, which has been limited by the low number of eukaryotic genomes available at the time, we are now able to provide a comprehensively analysis of the eukaryotic pathway evolution. The results show there has seemingly been a selection for the co-existence of NamPRT and NNMT in Deuterostomia, while the pathway dominant in bacteria is lost. This was accompanied by a marked increase in the number of NAD-consuming signalling enzymes. To explain these counterintuitive results we built a mathematical model of the pathway and demonstrated that NNMT has a critical role to maintain high NAD-consuming signalling fluxes by preventing accumulation of inhibitory Nam. Our model furthermore predicts that NNMT likely exerted an evolutionary pressure on the NamPRT affinity development. Simulating the resource competition we furthermore show that the presence of high affinity NamPRT together with NNMT makes the NADA dependent pathway obsolete. Based on multiple sequence alignments, we identified a sequence insertion in NamPRT in Deuterostomes that is affecting the affinity of NamPRT, undermining the predictions derived from our mathematical modelling approach

Taken together, our analyses suggest that the co-existence of NamPRT and NNMT has been a prerequisite to enable the evolutionary development of versatile NAD-dependent signalling mechanisms present in vertebrates.

%%% TeX-master: "manuscript"
% !TeX spellcheck=en_GB

\section{Results}

\subsection{Phylogenetic analysis of NAD biosynthesis and consumption}

NAD can be synthesised using several routes from three main precursors: tryptophan, nicotinamide (Nam) and nicotinic acid (NA) (Overview see fig.~\ref{fig:pathway_overview}). Nam and NA are together known as vitamin B3 or niacin. Alternatively, nicotinamide ribose (NR) can be used omitting the energetically unfavourable reaction of Nam phosphoribosyltransferase (NamPRT), requiring nicotinamide ribose kinase (NRK) instead \cite{Yoshino2018}. Due to the high turnover of NAD observed and the fact that only 1\% of the tryptophan taken up with our diet is converted into NAD\todo{source}, vitamin B3 (Nam and NA) and to a lower extend NR are the major precursors of human NAD biosynthesis.

Looking across all known species, two different pathways exist that recycle Nam. The major pathway found in yeast and plants uses a four-step pathway starting with the deamination of Nam to nicotinic acid by NADA. The other three enzymes comprise the Preiss-Handler pathway that also exists in vertebrates. The recycling pathway found in mammals directly converts Nam into the corresponding mononucleotide (NMN) a reaction catalysed by NamPRT and driven by a non-stoichiometric ATP-hydrolysis. A similar reaction catalysed by an evolutionary related enzyme NAPRT, converts NA into the NA mononucleotide in the Preiss-Handler pathway. NMN and NAMN are converted into dinucleotides by the Nam/NA adenylytransferases (NMNATs). The recycling pathway via NA finally requires an amidation step catalysed by NAD synthase, driven by the conversion of ATP to AMP producing pyrophosphate. Even though the latter pathway seems to be very inefficient, it is the pathway preferentially used by most bacteria, fungi and plants (see fig.~\ref{fig:phylo_distribution}A), whereas most metazoans recycle Nam using NamPRT.

Analysing the phylogeny of the NAD recycling enzymes in Metazoa in more detail reveals that not only NamPRT replaces NADA, but in most organisms, especially in Deuterostomia, NamPRT is found together with the Nam methyltransferase NNMT (fig.~\ref{fig:phylo_distribution}B). NNMT methylates Nam to methyl-Nam that is in mammals excreted with the urine, thus removing Nam from recycling. NNMT seems to have arisen \textit{de novo} in the common ancestor of Ecdysozoa and Lophotrochozoa. We were not be able to find any gene with considerable similarity in fungi or plants, even though in fungi genes named NNMT can be found in databases. These genes do, however, show very limited sequence similarities to the NNMT of nematodes or deuterostomes. The yeast protein has later been shown to be a lysin-protein-methyltransferase \cite{Wlodarski2011}. Nematodes are the only organisms that encode NNMT together with NADA without NamPRT being present. In Deuterostomia the only large class that does only have NamPRT and seems to have lost NNMT again are Sauropsida and here especially birds. The reason why a lot of birds do not encode NNMT remains unclear, as the appearance is quite scattered (figure S2\todo[author=Mathias B.]{Has to be made!}). The lack of NNMT might be related to the excretion system, as the product of NNMT methyl-Nam is in mammals excreted with the urine. There are some species where we could not find NamPRT or NADA but NNMT, we assume that this is due to incomplete genomes in the database.

In addition to the phylogenetic distribution of the two Nam salvage enzymes NADA and NamPRT, we looked at phylogenetic diversity of enzymes catalysing NAD-dependent signalling reactions. To do so we used the previously established classification into ten different enzyme families \cite{Gossmann2012FEBS}. The detailed list of templates used for the phylogenetic analysis can be found in supplementary table~S1. The numbers in figure~\ref{fig:phylo_distribution}B denote the average number of NAD-dependent signalling enzyme families found in each taxa. With the exception of Cnidaria and Lophotrochozoa, we find average three to four families in Protostomia, whereas most Deuterostomia have on average more than eight families with an increasing diversification of enzymes within some of these families \cite{Gossmann2014DNAR}.

Taken together, we found that the presence of NamPRT and NNMT coincides with an increased diversification of NAD-dependent signalling, whilst NADA is lost in vertebrates. This seems counterintuitive, as one would expect that a decrease in precursor concentration caused by the precursor removal through NNMT, should cause a decrease of NAD availability and consequently less active NAD dependent signalling.


\subsection{Dynamics of NAD biosynthesis and consumption}

So why does the diversity of NAD-dependent signalling increase? And why does NADA disappear in Deuterostomia although it is the predominant pathway in bacteria, plants and fungi? Given the complexity of the NAD metabolic network, these questions are difficult to be comprehensively addressed experimentally. We therefore built a dynamic model of NAD metabolism using existing kinetic data from the literature (for details see experimental procedures and supplementary table~S2).

To be able to compare metabolic features of evolutionary quite different systems in our simulations, and as we have limited information about expression levels of enzymes or changes of kinetic constants during evolution, we initially used the kinetic constants found for yeast or human enzymes for all systems analysed and used equal amounts of enzymes for all reactions. Wherever possible, we did not only include substrate affinities but also known product inhibitions or inhibition by downstream metabolites. As we assume that cell growth is, besides NAD-consuming reactions, a major driving force for NAD biosynthesis, we analysed different growth rates (cell division rates) by simulating different dilution rates for all metabolites.

First, we addressed the question of the predominant co-existence of NamPRT and NNMT in vertebrates that coincides with an increase in the number of NAD-consuming enzyme families. We thus used our dynamic model to simulate steady state concentrations and NAD consumption rates simulating NAD biosynthesis via NamPRT in the presence or absence of NNMT. We see that the presence of NNMT enables higher NAD consumption rates (fig.~\ref{fig:NNMT_NAD_flux}A) while, as expected, reducing the steady state concentration of NAD (fig.~\ref{fig:NNMT_NAD_flux}B). The decline in NAD concentration can be compensated by a higher expression of NamPRT, further increasing NAD consumption flux (dashed lines in fig.~\ref{fig:NNMT_NAD_flux}A and~B).

These findings can be explained when looking in more detail at the kinetic parameters of NamPRT and NAD-consuming enzymes such as Sirtuin~1. The increase of NAD consumption flux is caused by the fact that most NAD-consuming enzymes are inhibited by their product Nam, explaining why the presence of NNMT enables higher NAD consumption fluxes. At the same time, the high substrate affinity of NamPRT maintains a sufficiently high NAD concentration.

As kinetic parameters of NamPRT are only available for the human enzyme \cite{Burgos2008}, we went on to theoretically analyse the potential effect of NamPRT affinity ($K_{M}$) on NAD steady state concentration and NAD consumption flux. In the absence of NNMT (fig.~\ref{fig:NamPRT_affinity_Nam}A and~B) a change in $K_{M}$ has very little effect on steady state NAD concentration and NAD consumption flux. In the presence of NNMT, however, NAD consumption flux and NAD concentration increases with decreasing $K_{M}$ values (fig.~\ref{fig:NamPRT_affinity_Nam}C and~D).

It is interesting to note that without NNMT, NAD concentration and consumption flux are both considerably affected by cell division rates, at least if the enzyme expression is kept constant. This is, of course, an artificial scenario, as one would assume organisms to regulate enzyme expression to achieve similar levels of metabolite concentrations instead. However, there seems to be an apparent trade off between maintainable NAD concentration and consumption flux in the absence of NNMT. In contrast,  in the presence of NNMT NAD consumption rates increase with NAD concentration and both are relatively independent of cell division rates. \todo{The last sentence is unclear to me. Consumption increases with concentration? That is not in the figure. I needed to read often to understand what "NNMT and both" means.}

When we compare NAD consumption and NAD concentration with and without NNMT with two different substrate affinities of NamPRT, we see that at low affinity ($K_{M}$ of 1 $\mu$M, which is in the range of the $K_{M}$ of NADA for Nam), NAD consumption flux is only higher with NNMT at low cell division rates. This effect becomes stronger with decreasing affinity (fig.~\ref{fig:NamPRT_affinity_Nam}E and~F).

The pathway dynamics are not solely dependent on one enzyme. Thus, what is the impact of the substrate affinity of NNMT that is competing with NamPRT for the same substrate? In figure~\ref{fig:optimal_substr_affinities} we see that the substrate affinity values found in the human enzymes (indicated by black asterisks) are actually optimal with respect to both achievable steady state NAD concentration and consumption fluxes. Thus, a further increase of the affinity of NamPRT for Nam would not provide any advantage.


\subsection{Sequence variance acquired in metazoans enhances substrate affinity}

As our simulations suggest that NNMT might have exerted an evolutionary pressure on the development of NamPRT, we created a multiple sequence alignment to see if we can find sequence variations in the protein sequence of NamPRT that indicate evolutionary changes of NamPRT upon the appearance of NNMT. The multiple sequence alignment of selected sequences is shown in figure~\ref{fig:unresolved_loop}A and a more comprehensive multiple sequence alignment containing a larger number of species can be found in supplementary figure~S1. We recognised that most Deuterostomia that have only NamPRT and NNMT (indicated by the blue circle) have an insert of ten amino acids corresponding to positions 42 to 51 of the human enzyme. This insert overlaps with a predicted weak nuclear localisation signal, that is lost when removing this sequence stretch. The inserted sequence corresponds to a structurally unresolved loop in all available crystal structure of human NamPRT (e.g. \cite{Wang2006} structure visualisation fig.~\ref{fig:unresolved_loop}B). The loop is connected to one of the $\beta$-sheets involved in substrate binding.

From these observations, we derived two possible hypotheses regarding the role of the loop for NamPRT function. Firstly, we analysed whether the deletion of the amino acids 42 to 51 has an effect on the localisation of the human enzyme. We thus performed immunofluorescence imaging with FLAG-tagged proteins, but both wildtype and mutant protein showed a mixed cytosolic nuclear localisation (fig.~\ref{fig:unresolved_loop}C). We therefore conclude that the partial nuclear localisation of NamPRT is not affected by the sequence deletion.

The second hypothesis was that the sequence insertion influences NamPRT affinity, which is what our mathematical simulations would predict. We thus recombinantly expressed both wildtype and mutant NamPRT in \textit{E.~coli}. Upon purification, we measured the activity using NMR detection of NMN produced in the presence and absence of ATP. It appears that the enzymatic activity of the mutant enzyme is much lower than that of the wildtype enzyme (fig.~\ref{fig:unresolved_loop}D) and that an increase of the substrate concentration has a much stronger effect on the activity of the mutant compared to the wildtype enzyme, indicating a lower affinity. We furthermore showed that in contrast to NamPRT wildtype, the mutant does not exhibit an increased activity in the presence of ATP (fig.~\ref{fig:unresolved_loop}E).


\subsection{Loss of NADA in vertebrates}

The remaining question is now, why NADA is lost in vertebrates, while NADA and NamPRT are co-existing in bacteria. To analyse this question we built a two compartment model that has a shared Nam source (details see experimental procedures and supplementary table~S2). One compartment contains NADA while the other contains either NamPRT alone at equal expression levels, or together with NNMT. Without NNMT the compartment containing NADA has slightly lower NAD consumption rates (fig.~\ref{fig:NNMT_comp_advantage}A), but is able to maintain much higher steady state NAD concentrations at low cell division rates (fig.~\ref{fig:NNMT_comp_advantage}B). At high cell division rates, steady state concentrations in both compartments are similar. The latter might explain why in bacteria that have relatively high growth rates both systems co-exist.

In the presence of NNMT, the NamPRT compartment out-competes the NADA-containing compartment, both with respect to NAD consumption rates and steady state NAD concentrations (fig.~\ref{fig:NNMT_comp_advantage}C and~D). This effect is dependent on a high affinity of NamPRT, as at low affinity (high $K_{M}$), the NADA containing system is again able to maintain higher NAD concentrations. The NAD consumption flux is still higher in the NamPRT and NNMT containing compartment. Taken together this might explain why upon the development of a sufficiently high affinity of NamPRT for its substrate, which seems to have been induced by the appearance of NNMT, NADA is lost.

%%% TeX-master: "manuscript"
% !TeX spellcheck=en_GB

\section{Discussion}

We here comprehensively analysed the phylogenetic distribution of the three enzymes using Nam as a substrate. These are the two NAD salvage pathway enzymes NADA and NamPRT as well as the Nam-degrading enzyme NNMT. We found that after the first appearance of NNMT in Protostomia, a diversification of NAD-consuming reactions in Deuterostomia can be observed. We could explain these finding using mathematical modelling, as NNMT removes excess Nam from cells and thereby reduces product inhibition of NAD signalling enzymes. This in turn enables higher fluxes through these reactions. Thus, the diversification of NAD-consuming enzymes in mammals seems to have been enabled by the presence of NNMT.

NAD-consuming enzymes are involved in a wide variety of signalling and gene regulatory mechanisms that, due to their sensitivity to NAD$^{+}$, have the ability to translate differences in metabolic states into changes in signalling and gene regulation. As NAD concentrations are lowered by the removal of NAD precursor by NNMT, a high-affinity enzyme for recycling of Nam is required to ensure maintenance for cellular NAD levels for other metabolic processes. It is unclear, why NamPRT developed a higher affinity and not NADA. NADA may have thermodynamic or mechanistic limitations or other environmental conditions may have been responsible for the observed evolutionary trajectory.  Still, our simulations clearly show that a high affinity of NamPRT is required for high NAD consumption fluxes and NAD concentrations. It therefore seems plausible that NNMT might have been driving NamPRT evolution. Looking at the enzyme affinities of the human enzymes it furthermore appears that both NNMT and NamPRT reached an almost optimal state, as further changes in the affinity of NamPRT or NNMT would not result in much higher steady state NAD concentrations or NAD consumption fluxes, according to our simulations. In addition, the data suggest that NNMT makes both NAD concentration and NAD consumption relatively independent of other processes requiring NAD, such as cell growth.

Our findings shed also new light on the potential physiological role of NNMT, which has earlier been recognized as potential marker for some types of cancer (e.g. \cite{Okamura1998}). The major healthy tissue expressing NNMT is the liver, while no or only little expression of NNMT is observed in most other tissues \todo{ref}. Increased NNMT expression in cancer might serve to remove Nam produced by increased NAD-dependent signalling. To maintain high NAD concentrations, a simultaneous higher expression of NamPRT is required, which is what has been found in some types of cancer \cite{Bi2011,Wang2011}. NNMT is only advantageous as long as NamPRT affinity is sufficiently high. This suggests that certain types of cancer that express NNMT at a high level would be more susceptible to competitive inhibitors of NamPRT. Several of such inhibitors are currently tested in clinical studies \cite{Espindola-Netto2017} \todo{Are there more refs?}. Based on our analysis, we would suggest that it might be reasonable to screen patients before treatment, as non-NNMT expressing tumours might respond less to competitive NamPRT inhibitors and missing Nam degradation in those cancer cells would potentially lead to an accumulation of Nam that could outcompete the inhibitor. The latter aspect is unclear and requires further investigation.

Our combined phylogenetic/modelling analysis provides a potential explanation both for the co-occurrence of NADA and NamPRT in bacteria and for the loss of NADA in vertebrates. We here show that when two compartments compete for the same limited source of Nam, the compartment that contains NamPRT and NNMT has a higher steady state NAD concentration and NAD consumption rate than the compartment containing NADA. The dominant enzyme combination found in vertebrates, a high-affinity NamPRT with NNMT, seems to provide a competitive advantage. As this may also hold for mammalian-associated bacteria, particularly pathogens, we wanted to see whether pathogenic bacteria solely express NamPRT. Unfortunately, bacterial habitat information is currently far from complete and often difficult to access. We therefore manually checked bacteria that possess NamPRT and indeed found that most of them have been characterised to be pathogenic. It should be noted that the distribution of NADA and NamPRT does not follow the bacterial species tree but is rather scattered \cite{Gazzaniga2009}. This distribution hints to a functional importance of posssessing NADA or NamPRT that might be due to different environments.

A detailed analysis of sequence variances in NamPRT revealed that only deuterostomes that have NNMT but not NADA, have a sequence insertion in the N-terminal part of NamPRT that seems to enable the high affinity of the enzyme. This in turn would suggest that also the bacterial enzymes do not have a high substrate affinity. The only kinetically characterised NamPRT is from \textit{Acinetobacter baylyi} \cite{Sorci2010} and has a $K_{M}$ for Nam that is about 10\,000 times higher than the $K_{M}$ of human NamPRT, supporting our hypothesis. Other bacterial NamPRTs were shown to be functional \cite{Martin2001,Gerdes2006}, but no kinetic parameters were measured. The differences in activity and affinity, implying differences in substrate binding could potentially be exploited for the development of antibiotics. Further analysis possibly including the crystallisation of a bacterial NamPRT would be required, to see whether the bacterial NAD metabolism could be a promising target.

% Martin2001: Haemophilus ducreyi: no kinetic measurements
% Sorci2010: Acinetobacter sp.: NamPRT (Nam) Km: 0.04 mM (= 40 000 nM); kcat: 0.12/s (apparent values determined at constant 5 mM PRPP and 2 mM ATP)
% Gerdes2006: Synechocystis sp.: only activity 0.5 U/mg NamPRT for Nam
% Human, according to table S2: Km: 5 nM; kcat: 0.0077/s

In our analysis, we completely neglected the potential effects of co-substrates of the investigated pathway. Neglected co-substrates include targets of the NAD-consuming enzymes, such as acylated proteins for sirtuins, for example, or phosphoribosyl pyrophosphate (PRPP) and ATP that are required for NMN synthesis by NamPRT. Also the presence of the methyl donor \textit{S}-adenosyl methionine (SAM) and its precursor methionine that have been shown to potentially limit the effect of NNMT \cite{Ulanovskaya2013} was not considered. All of these co-substrates should be included in future analyses, but information about the \textit{in~vivo} concentration of those co-substrates is currently very limited. We anticipate improvements as more sensitive methods for metabolite measurements are currently developed \todo{ref? Or how do you know?}.

% todo: The following three paragraphs were not proofread, yet.

During our analysis we came across several problems related to the NCBI sequence database. One is sequence contamination, which is a well known problem \todo{ref?} and we therefore tried to remove all sequences of obvious bacterial origin from the analysis, based on sequence homology analysis. Another problem is incomplete genomes. Although there are tools to assess the completeness of a genome (e.g.~\cite{Simao2015}), none of them could convincingly claim to be reliable. The genomes of the common model organisms can probably be assumed to be close to complete, but there are many draft genomes in the databases whose completeness is uncertain. Even if the completeness would be known, with the high number of genomes used in this analysis, it is likely that some genes of interest were not sequenced in every genome. For our analysis, this means that scattered patterns of few missing genes may be real or stem from an incomplete genome.

The third problem are wrong annotations. We tried to avoid these, by only relying on template sequences with confirmed function, wherever possible. This problem becomes apparent by the fact that in yeast an enzyme named NNMT can be found, which is based on an initial analysis on life span extension in \textit{Saccharomyces cerevisae} \cite{Anderson2003}. The protein has, however, later been shown not function as methyltransferase for Nam but for the eukaryotic elongation factor~1A (eEF1A) giving it its new name elongation factor methyltransferase~7 (Efm7) \cite{Hamey2016}. The old name is still present in many databases, though.

Nevertheless, we have been able to comprehensively analyse the functional co-evolution of several enzymes of the NAD pathway with the appearance of NNMT seemingly initiating and driving complex alterations of the pathway such as an increase and diversification of NAD-dependent signalling, followed by an increase in NamPRT substrate affinity (schematic overview see fig.~\ref{fig:evo_events}). This again appears to be accompanied by the loss of NADA in vertebrates and the first gene duplication of NMNATs \cite{Lau2010}. We also noted that the second gene duplication of NMNATs and thus the further compartmentalisation of NAD metabolism is co-occurring with a site-specific positive selection event in NNMT (unpublished results). To our knowledge, this is the first study that comprehensively demonstrates a interdependency of enzyme evolution based on functional interaction of several enzymes of a pathway. While it is of course well know and described for directly interaction proteins \todo{Not sure that is true, one that might come close is the following work PMID:26446903 \cite{Figliuzzi2016}. Please check and potentially revise statement}.

%%% TeX-master: "manuscript"

\section{Figure Legends}

\subsection{Figure 1}

\textbf{Schematic overview of NAD biosynthesis and consumption.} NAD can be synthesized from tryptophan (Trp), nicotinamide (Nam), nicotinic acid (NA), and, to a lesser extend, nicotinamide ribose (NR). Nam is the main precursor and also the product of NAD-consuming signalling reactions by enzymes such as sirtuins (NAD-dependent histone deacetylases) or PARPs (poly-ADP-ribosylases). For the recycling of Nam, two different pathways exist. The pathway found in yeast and many bacteria starts with the deamination of Nam by Nam deamidase (NADA). The other three enzymes comprise the Preiss-Handler pathway that also exists in vertebrates. The pathway found in vertebrates directly converts Nam into the corresponding mononucleotide (NMN) by the Nam phosphoribosyltransferase (NamPRT). A third enzyme can consume Nam, the Nam N-methyltransferase (NNMT). For more details and abbreviations, see text.


\subsection{Figure 2}

\textbf{Evolutionary distribution of NADA, NNMT and NamPRT and their relation to the number of NAD consumers.} A) Distribution of NADA, NNMT and NamPRT in selected major taxa. NADA is dominant in Bacteria, Fungi, and Plants (Viridiplantae), whereas NamPRT together with NNMT is dominant in Metazoa. Numbers at the pie charts show, how many species of the taxon possess the respective enzyme combination indicated by the colour explained in the lower right of the figure. Below the taxon name, the number of species in that taxon is given. B) Common tree of selected taxa within the Metazoa, including 334 species. The pie charts indicate the distribution of species within the respective taxon that have the enzyme combination indicated by the colour explained in the lower right. The size of the pie charts is proportional to the logarithm of the number of species analysed in the particular taxon. The numbers below the taxon names indicate the average number of NAD-consuming enzyme families found in all sub-taxa. The branch length is arbitrary.


\subsection{Figure 3}

\textbf{NNMT enables high NAD-consumption flux.} We used a dynamic model of NAD biosynthesis and consumption (details see Materials and Methods) to simulate NAD consumption flux (A) and NAD concentration (B) in the presence of NamPRT and with or without NNMT at different cell division rates. NMMT enables higher steady-state NAD consumption flux despite reduced NAD concentrations.


\subsection{Figure 4}

\textbf{Role of NamPRT affinity for Nam.} Using the dynamic model of NAD biosynthesis and consumption we simulated the affect of different Michaelis Menten constants ($K_M$) of NamPRT for Nam on steady-state NAD consumption flux and NAD concentration at different cell devision rates. In the absence of NNMT (A and B), the $K_M$ of NamPRT has little influence on NAD consumption and concentration, but both are changing with cell devision rates. In the presence of NNMT (C and D), decreasing $K_M$ of NamPRT enables increasing NAD consumption flux and NAD concentration. NNMT furthermore makes both, consumption flux and concentration, relatively independent of cell division rates. Comparing the situation with and without NNMT (E and F) at different NamPRT $K_M$ reveals that at low $K_M$ and high cell devision rates NNMT no longer enables higher NAD consumption rates compared to NamPRT alone.


\subsection{Figure 5}

\textbf{The substrate affinities of human NNMT and NamPRT are optimal.} Both NAD consumption flux (A) and NAD concentration (B) are increasing with decreasing $K_M$ of NamPRT but decreeasing $K_M$ of NNMT. The affinities reported for human enzymes (indicated by a black asterisk) appear to be close to optimal, as further improvements would have little or no effect on NAD consumption or concentration.


\subsection{Figure 6}

\textbf{The function of the structurally unresolved loop structure of NamPRT.} Most Deuterostomia that encode NNMT show a sequence insertion in the N-terminal region of NamPRT that has been revealed by multiple sequence alignment of NamPRT from different species (A). Coloured circles indicate the enzymes present in the species besides NamPRT; blue: NNMT; black: NADA and NNMT; yellow: NADA. B) The visualisation of human NamPRT is based on a structure prediction of SWISS-MODEL \cite{Arnold2006,Biasini2014} of the sequence of the human NamPRT using the model 2H3D as template \cite{Wang2006}. The inserted region is not resolved in crystal structures of NamPRT and thus appears to be a flexible loop structure at the surface of the NamPRT dimer, coloured in red. C) The localisation of the FLAG-tagged mutant protein lacking the unresolved loop is not changed compared to wildtype human NamPRT. Both show a heterogenous nuclear cytosolic localisation in immunofluerescence images in HeLa S3 cells. But \textit{in vitro} measurements using recombinant protein show that the mutant NamPRT has lower activity than the wildtype enzyme (D) and is not activated by ATP (E).


\subsection{Figure 7}

\textbf{NNMT provides a competitive advantage and makes NADA obsolete.} To simulate competition for common resources, we created a two compartment model where one compartment contained NADA but no NamPRT and the other compartment contained NamPRT either with or without NNMT but no NADA. In the absence of NNMT (A and B) the compartment containing NADA has slightly lower NAD consumption rates (A) but much higher NAD concentrations (B). In the presence of NNMT, however, both NAD consumption (C) and NAD concentration (D) are lower in the NADA compartment, but this effect is only observed at low NamPRT $K_M$.


\subsection{Comments and parts of old version not included yet}

Chordata: NADA disappears. These analyses also support the view that Tunicata and Branchiostoma are not part of the phylum chordata\todo[author=Mathias]{Is this view in any way argued upon by anyone??}.

In the evolutionary context, an additional question arises: Why do only a few organisms, mostly less complex animals, possess the gene for NADA in addition to NamPRT and NNMT. When including NADA into the simulations, we see that the effect of NADA on NAD consumption is very limited in the presence of NNMT (old Figure 4F) even at high expression levels (see old Suppl. Figure S1C and D) and could be compensated by increased expression of NamPRT (not shown).

Until this point we have neglected compartmentalisation of the pathway. We do however know from previous studies that in early vertebrate development a compartmentalisation of the pathway has occurred reflected by a gene triplication of NMNAT and the occurrence of compartment-specific domains called ISTIDs (Lau, 2010). Looking at the evolutionary timepoint \todo[author=Toni]{This time point can only be reconstructed, but is unknown. I think a figure with the likely reconstruction would help to follow the argumentation} of appearance of NNMT and the gene triplication of NMNATs, we see that NNMT occurs prior to the gene triplication and the first occurrence of ISTIDs and we have confirmed this in our own analysis (not shown).





\end{document}
