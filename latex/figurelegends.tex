%%% TeX-master: "manuscript"

\section{Figure Legends}

\subsection{Figure 1}

Schematic overview of NAD biosynthesis and consumption. NAD can be synthesized using several routes from altogether three main precursors: tryptophan (Trp), nicotinamide (Nam), and nicotinic acid (NA). Nam and NA are together known as vitamin B3 or niacin. To a lesser extend, nicotinamide ribose (NR) can be used omitting the energetically unfavourable reaction of Nam phosphoribosyltransferase (NamPRT), requiring nicotinamide ribose kinase (NRK) instead. Only 1\% of the tryptophan taken up with our diet is converted into NAD, thus, vitamin B3 and, to a lower extend, NR are essential components of our diet, with Nam being the major NAD precursor in humans\todo{Whole sentence (starting with "Only") not necessarily part of the legend. What do you think, Ines?}. Nam is furthermore the product of NAD-consuming signalling reactions such as sirtuins (NAD-dependent histone deacetylases) or PARPs (poly-ADP-ribosylases).

For the recycling of Nam, two different pathways exist. The pathway found in yeast and many bacteria is using a four-step pathway starting with the deamination of Nam to nicotinic acid (NA) by Nam deamidase (NADA). The other three enzymes comprise the Preiss-Handler pathway that also exists in vertebrates. The recycling pathway found in vertebrates directly converts Nam into the corresponding mononucleotide (NMN) a reaction catalysed by NamPRT and driven by a non-stoichiometric ATP-hydrolysis. A similar reaction catalysed by an evolutionary related enzyme NAPRT, converts NA into the NA mononucleotide (NAMN). NMN and NAMN are converted into dinucleotides by the Nam/NA adenylytransferases (NMNATs). The recycling pathway via NA finally requires an amination step catalysed by NADsynthase (NADS), driven by the conversion of ATP to AMP producing pyrophosphate.


\subsection{Figure 2}

Evolutionary distribution of NADA, NNMT and NamPRT and their relation to the number of NAD consumers. A) Distribution of NADA, NNMT and NamPRT in selected major taxa. NADA is dominant in Bacteria, Fungi, and Plants (Viridiplantae), whereas NamPRT together with NNMT is dominant in Deuterostomia. Numbers at the pie charts show, how many species of the taxon possess the respective enzyme combination indicated by the colour explained in the lower right of the figure. Below the taxon name, the number of species in that taxon is given.

B) Common tree of selected taxa within the Metazoa, including 334 species. The pie charts indicate the distribution of species within the respective taxon that have the enzyme combination indicated by the colour, explained in the lower right. The size of the pie charts is proportional to the logarithm of the number of species analysed in the particular taxon. The numbers below the taxon names indicate the average number of NAD-consuming enzyme families found in all sub-taxa. The branch length is arbitrary.


\subsection{Figure 3}

NAD consumption flux and NAD concentration in simulations of organisms with NADA(?) and with and without NNMT at different Nam import and cell division rates.


\subsection{Figure 4}

Impact of NNMT on NAD consumption and NAD concentration in simulations of organisms with NADA or NamPRT.


\subsection{Figure 5}

NAD consumption flux and NAD concentration in simulations of organisms with NamPRT and with and without NNMT at different KMs for NamPRT and cell division rates.
%%% TeX-master: "manuscript"


\subsection{Figure 6}

The substrate affinity of NNMT and NamPRT have opposite effects on NAD consumption (A) and concentration (B), as would be expected. The affinities previously measured for human enzymes (indicated by a black asterisk) appear to be close to optimal, as further improvements would have little or no effect on NAD consumption or concentration.


\subsection{Figure 7}

Deuterostomia that encode NNMT show a sequence insertion in the N-terminal region of NamPRT. A) Multiple sequence alignment of NamPRT of selected deuterostomes show a sequence insertion in organisms that encode NamPRT and NNMT. Coloured circles indicate the enzymes present in the species; blue: NamPRT and NNMT; black: NamPRT, NADA, and NNMT; yellow: NamPRT and NADA.

B) The inserted region is not resolved in crystal structures of human NamPRT and thus appears to be a flexible loop structure at the surface of the NamPRT dimer, coloured in red. The visualisation is based on a structure prediction of SWISS-MODEL \cite{Arnold2006,Biasini2014} of the sequence of the human NamPRT (P43490) using the model 2H3D as template \cite{Wang2006}.
