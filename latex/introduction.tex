%%% TeX-master: "manuscript"
% !TeX spellcheck=en_GB

\section{Introduction}

NAD metabolism represents one of the most critical links that connect cellular signal transduction and energy metabolism. Even though best known as cofactor for various redox-reactions , NAD is involved in a number of signalling processes that consume NAD$^{+}$ by cleaving the molecule to nicotinamide (Nam) and ADP-ribose \cite{Verdin2015}. These NAD-dependent signalling reactions include but are not limited to poly- and mono-ADP-ribosylation, NAD-dependent protein deacylation by sirtuins as well as the synthesis of calcium-mobilizing molecules such as cyclic ADP-ribose \cite{?}. These NAD-dependent signalling processes participate in the regulation of virtually all cellular activities. The enzymes involved in these processes are sensitive to the available NAD concentration \cite{Ruggieri2015}, which in turn is dependent on the NAD$^{+}$/NADH redox ratio. Therefore, NAD-dependent signalling can act as a transmitter of changes in the cellular energy homeostasis, for example, to regulate gene expression or metabolic activity \cite{Koch-Nolte2009}.\todo[author=Mathias B.]{Please add more references to the first paragraph. I think there are better references then my own review.}

The significance of NAD-dependent signalling for NAD homeostasis has long been underestimated. However, it has now been established that substances affecting NAD biosynthesis lead to a rapid decline of the NAD concentration \cite{Buonvicino2018} suggesting that NAD-dependent signalling consumes substantial amounts of NAD, which is why we later refer to them also as NAD-consuming reactions. The resulting NAD turnover differs in a cell type specific manner and can lead to an NAD half-life as short as 2 hours \cite{Liu2018}. To maintain the NAD concentration at physiological levels, NAD biosynthesis needs to act at an equally rapid rate. Imbalances in NAD homeostasis have been linked to various, in particular, ageing-related diseases such as diabetes, neurodegenerative disorders and cancer \cite{Chiarugi2012,Verdin2015}. Several recent studies have demonstrated impressive health benefits of dietary supplementation with intermediates of NAD biosynthesis including NMN and nicotinamide riboside, (NR) \cite{Yoshino2018}. Apparently, the exploitation of NAD biosynthetic routes, in addition to the use of nicotinamide as precursor (Fig. 1), results in increased NAD concentrations that stimulate NAD-dependent signalling processes, in particular, protein deacetylation by sirtuins \cite{North2004}.

Due to the constant release of Nam through NAD-consuming signalling reactions, the NAD salvage pathway using Nam as precursor is the most important NAD synthesis pathway. If Nam would not be constantly recycled into NAD, humans would require a much higher daily vitamin B3 intake than the 16 mg that are the current daily recommendation \cite{CommissionofEuropeanCommunities2008}. In the first step of the salvage pathway, Nam is converted to the mononucleotide, NMN, by Nam phosphoribosyltransferase (NamPRT) using phosphoribosyl pyrophosphate (PRPP) as co-substrate. The nearly complete recycling of Nam is achieved by an extraordinary high affinity of NamPRT to Nam, the $K_{M}$ being in the low nanomolar range \cite{Burgos2008}. Despite the importance of its salvage, Nam can also be marked for excretion by methylation. The presence of nicotinamide N-methyltransferase (NNMT) in vertebrates \cite{Gossmann2012FEBS} is among the most enigmatic and counterintuitive features of NAD metabolism. Why is there one enzyme (NamPRT) seemingly optimised to recycle even the faintest amounts of Nam back into NAD synthesis, while at the same time there is another enzyme (NNMT) that seems to have no metabolic function other than to remove Nam from NAD metabolism, This puzzle becomes even more intriguing when considering that the majority of lower organisms and plants deamidated Nam to nicotinic acid (NA) using the nicotinamide deamidase (NADA) before it can enter NAD biosynthesis via the Preiss-Handler pathway (Fig. 1).

We here present a phylogenetic analysis of NAD pathway in eukaryotes that. In contrast to our previous analysis, which has been limited by the low number of eukaryotic genomes available at the time, we are now able to provide a comprehensively analysis of the eukaryotic pathway evolution. The results show there has seemingly been a selection for the co-existence of NamPRT and NNMT in Deuterostomia, while the pathway dominant in bacteria is lost. This was accompanied by a marked increase in the number of NAD-consuming signalling enzymes. To explain these counterintuitive results we built a mathematical model of the pathway and demonstrated that NNMT has a critical role to maintain high NAD-consuming signalling fluxes by preventing accumulation of inhibitory Nam. Our model furthermore predicts that NNMT likely exerted an evolutionary pressure on the NamPRT affinity development. Simulating the resource competition we furthermore show that the presence of high affinity NamPRT together with NNMT makes the NADA dependent pathway obsolete. Based on multiple sequence alignments, we identified a sequence insertion in NamPRT in Deuterostomes that is affecting the affinity of NamPRT, undermining the predictions derived from our mathematical modelling approach

Taken together, our analyses suggest that the co-existence of NamPRT and NNMT has been a prerequisite to enable the evolutionary development of versatile NAD-dependent signalling mechanisms present in vertebrates.
