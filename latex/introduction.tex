%%% TeX-master: "manuscript"

\section{Introduction}

NAD metabolism represents one of the most critical links that connect cellular signal transduction and energy metabolism.  Even though best known as cofactor for various redox-reactions , NAD is involved in a number of signalling processes that consume NAD+ by cleaving the molecule to nicotinamide (Nam) and ADP-ribose \cite{Verdin2015}. These NAD+-dependent signalling reactions include but are not limited to poly- and mono-ADP-ribosylation, NAD+-dependent protein deacylation by sirtuins as well as the synthesis of calcium mobilizing molecules such as cyclic ADP-ribose \cite{Opitz2015}. These NAD+-dependent signalling processes participate in the regulation of virtually all cellular activities. The enzymes involved in these processes are sensitive to the available NAD+ concentration \cite{Ruggieri2015}, which in turn is dependent on the NAD+/NADH redox ratio. Therefore, NAD+-dependent signalling can act as a transmitter of changes in the cellular energy homeostasis, for example, to regulate gene expression or metabolic activity \cite{Koch-Nolte2009}.

The significance of NAD+-dependent signalling for NAD homeostasis has long been underestimated. However, it has now been established that substances affecting NAD biosynthesis  lead to a rapid decline  of the NAD concentration \cite{Vacor2018}\todo[author=Ines]{Insert bibtex reference} suggesting that NAD+-dependent signalling consumes substantial amounts of NAD, that is why we later refer to them also as NAD-consuming reactions. The resulting NAD turnover differs in a cell type specific manner and can lead to an NAD-halflife as short as 2 hours  \cite{Rabinowitz2018}\todo[author=Ines]{Insert bibtex reference}. To maintain the NAD concentration at physiological levels, NAD biosynthesis needs to act at an equally rapid rate. Imbalances in NAD-homeostasis have been linked to various, in particular, ageing-related diseases such as diabetes, neurodegenerative disorders and cancer \cite{Chiarugi2012,Verdin2015}\todo[author=Ines]{Potentially adjust ref}. Severalrecent studies have demonstrated impressive health benefits of dietary supplementation with intermediates of NAD biosynthesis including NMN and nicotinamide riboside, NR \cite{ImaiCellmetab2018}\todo[author=Ines]{Insert bibtex reference}. Apparently, the exploitation of NAD biosynthetic routes, in addition to the use of nicotinamide as precursor (Fig. 1), results in increased NAD concentrations that stimulate NAD+-dependent signalling processes, in particular, protein deacetylation by sirtuins \todo[author=Mathias B.]{Please add references}.

In mammals, NAD biosynthesis predominantly relies on nicotinamide \todo[author=Mathias Z.]{Please add references}, a form of vitamin B3, as precursor. In the first step, Nam is converted to the mononucleotide, NMN, by Nam phosphoribosyltransferase (NamPRT) using phosphoribosylpyrophosphate (PRPP) as co-substrate. The key role of this enzyme is related to the nature of NAD+-dependent signalling reactions, namely, that in all these reactions Nam is cleaved off and potentially lost. If the released Nam were not recycled into NAD biosynthesis, humans would require a much higher daily vitamin B3 intake than the 15 mg that are the current daily recommendation \todo[author=Mathias Z.]{I have not been able to find a reference for that so far.}. The nearly complete recycling of Nam is achieved by an extraordinary high affinity of NamPRT to Nam, the Km being in the low nanomolar range \cite{Burgos2008}. Despite the importance of its salvage, Nam can also be marked for excretion by methylation. Indeed, the presence of nicotinamide N-methyltransferase (NNMT) in vertebrates \cite{Gossmann2012} is among the most enigmatic and counterintuitive features of NAD metabolism. Why is there one enzyme (NamPRT) seemingly optimized to recycle even the faintest amounts of Nam back into NAD synthesis, while at the same time there is another one (NNMT) that seems to have no metabolic function other than to remove Nam from NAD metabolism, This puzzle becomes even more intriguing when considering that the majority of lower organisms and plants deaminated Nam to nicotinic acid (NA) using the nicotinamide deamidase (NADA) before it can enter NAD biosynthesis via the Preiss-Handler pathway (Fig. 1).

We here present a phylogenetic analys of NAD-pathway in eukaryotes that  in contrast to our previous analysis ,that has been lmited by the low number of eularyotic genomes available at the time, now comprehensively analysis the evolution of the pathway. In the present study, we furthermore combine the phylogenetic analysis with a mathematical modelling approach. This has proven to be  a powerful tool to analyse the relationship between NAD biosynthetic and signalling fluxes when different subsets of Nam-converting enzymes are present. that enable us to explain the counterintuitive pathway evolution, such as that in Deuterostomes, there has been a strong tendency towards the co-existence of NamPRT and NNMT, while the pathway dominant in bacteria is lost in vertebrates. Moreover, this selection for the co-existence of NamPRT and NNMT was accompanied by a marked increase in the number of genes encoding NAD+-consuming signalling reactions.  The model demonstrated that NNMT has a critical role to maintain high NAD+-consuming signalling fluxes by preventing accumulation of inhibitory Nam. We furthermore show that the presence of high affinity NamPRT together with NNMT makes the NADA dependent pathway obsolete. We  furthermore identify a sequence insertion only found in Deuterostomes that encode NNMT but not NADA and experimentally verify that this insertion is indeed changing the affinity of NamPRT.
Taken together, our analyses suggest that the co-existence of NamPRT and NNMT has been a prerequisite to enable the evolutionary development of versatile NAD+-dependent signalling mechanisms present in vertebrates, and that NNMT likely exerted an evolutionary pressure on the NamPRT affinity development.
