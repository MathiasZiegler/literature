%%% TeX-master: "manuscript"
% !TeX spellcheck=en_GB

\section{Introduction}

NAD metabolism represents one of the most critical links that connect cellular signal transduction and energy metabolism. Even though it is best known as cofactor for various redox-reactions, NAD is involved in a number of signalling processes that consume NAD$^{+}$ by cleaving the molecule to nicotinamide (Nam) and ADP-ribose \cite{Verdin2015}. These NAD-dependent signalling reactions include poly- and mono-ADP-ribosylation \cite{Butepage2015,DeVos2012}, NAD-dependent protein deacylation by sirtuins \cite{Osborne2016}, and the synthesis of calcium-mobilizing molecules such as cyclic ADP-ribose \cite{Lee2012}. These NAD-dependent signalling processes participate in the regulation of virtually all cellular activities. The enzymes involved in these processes are sensitive to the available NAD concentration \cite{Ruggieri2015}, which in turn is dependent on the NAD$^{+}$/NADH redox ratio. Therefore, NAD-dependent signalling can act as a transmitter of changes in the cellular energy homeostasis, for example, to regulate gene expression or metabolic activity \cite{Koch-Nolte2009}.

The significance of NAD-dependent signalling for NAD homeostasis has long been underestimated. It has now been established, however, that substances affecting NAD biosynthesis lead to a rapid decline of the NAD concentration \cite{Buonvicino2018}. This suggests that NAD-dependent signalling reactions consume substantial amounts of NAD. Therefore, we later refer to them also as NAD-consuming reactions. The resulting NAD turnover differs in a cell-type-specific manner and can lead to an NAD half-life as short as two hours \cite{Liu2018}. To maintain the NAD concentration at physiological levels, NAD biosynthesis needs to act at an equally rapid rate. Imbalances in NAD homeostasis have been linked to various mainly age related diseases, such as diabetes, neurodegenerative disorders, and cancer \cite{Chiarugi2012,Verdin2015}. Several recent studies have demonstrated impressive health benefits of dietary supplementation with intermediates of NAD biosynthesis including Nam mononucleotide (NMN) and Nam riboside (NR) \cite{Yoshino2018}. Apparently, the exploitation of NAD biosynthetic routes, in addition to the use of nicotinamide as precursor (fig.~\ref{fig:pathway_overview}), results in increased NAD concentrations that stimulate NAD-dependent signalling processes, in particular, protein deacetylation by sirtuins \cite{North2004}.

Due to the constant release of Nam through NAD-consuming signalling reactions, the NAD salvage pathway using Nam as precursor is the most important NAD synthesis pathway. If Nam were not continuously recycled into NAD, humans would require a much higher daily vitamin B3 intake than the 16\,mg that are the current daily recommendation \cite{CommissionofEuropeanCommunities2008}. Two principal pathways exist that recycle Nam. Firstly, vertebrates use a direct two step pathway starting with the conversion of Nam into the mononucleotide NMN by the Nam phosphoribosyltransferase (NamPRT) using phosphoribosyl pyrophosphate (PRPP) as co-substrate. The nearly complete recycling of Nam by NamPRT is achieved by an extraordinary high substrate affinity to Nam, the $K_{M}$ being in the low nanomolar range \cite{Burgos2008}. This appears to be mediated by an ATP-dependent phosphorylation of a histidine residue in the catalytic core \cite{Burgos2009}. Despite the importance of its salvage, Nam can also be marked for excretion by methylation. The presence of nicotinamide N-methyltransferase (NNMT) in vertebrates \cite{Gossmann2012FEBS} is among the most enigmatic and counterintuitive features of NAD metabolism. While NamPRT is seemingly optimised to recycle even the faintest amounts of Nam back into NAD synthesis, NNMT seems to have no metabolic function other than to remove Nam from NAD metabolism. It has been suggested that the process potentially acts as a metabolic methylation sink \cite{Pissios2017}.

Secondly, in most prokayotes as well as in plants and fungi, a pathway consisting of four steps starting with the deamidation of Nam to nicotinic acid (NA) by the Nam deamidase (NADA) is used. (fig.~\ref{fig:pathway_overview}). The three enzymes that act after NADA belong to the Preiss-Handler pathway that also exists in vertebrates. NA is converted into the corresponding mononucleotide (NAMN), in a reactions performed by the NA-specific phosphoribosyltransferase NAPRT. The conversion of both mononucleotides, NMN and NAMN, into their corresponding dinucleotides, NAD and NAAD, is catalysed by the Nam/NA adenylytransferases (NMNATs) that are essential in all organisms \cite{DeFigueiredo2011}. The recycling pathway via NA finally requires re-amidation of NAAD by NAD synthase. This final reaction includes an enzyme adenylation step that consumes ATP. Therefore, the Nam recycling by NADA is energetically less efficient under normal conditions than the recycling pathway starting with NamPRT. 

We and others have earlier shown that the two pathway co-exists in some eukaryotes\todo{Insert Refs}, as well as in some bacterial species\todo{MB: please check in the respective bacterial artical if they reported the co-eixtence. I think they did.}.  But why we observe such a scattered distribution of the two pathways is still unknown. We furthermore have little understanding of the physiological role of NNMT and its impact on NAD-metabolism so far.


As earlier analysis have been limited by the few eukaryotic genomes available at the time, we here performed a comprehensive phylogenetic analysis of the NAD pathways using ... eukaryotic and ... prokaryotic genomes \todo{include numbers}. Our results suggest that there has been a selection for the co-existence of NamPRT and NNMT in deuterostomes, while the deamidation pathway, which is dominant in bacteria, is lost. This transition was accompanied by a marked increase in the number of NAD-consuming signalling enzymes. Mathematical modelling of the pathway revealed an unexpected positive kinetic role of NNMT in the maintenance of high NAD-consuming signalling fluxes, preventing accumulation of inhibitory Nam. In addition, the model predicts that NNMT likely exerted an evolutionary pressure on NamPRT to develop a high affinity towards its substrate Nam. Indeed, we identified a short sequence insertion in NamPRT, which first occurs in Deuterostomes and that appears to modulate the affinity of NamPRT. Simulating the resource competition, we furthermore show that the presence of high affinity NamPRT together with NNMT makes the NADA-dependent pathway obsolete.

Taken together, our analyses suggest that the co-existence of NamPRT and NNMT has been a prerequisite to enable the evolutionary development of versatile NAD-dependent signalling mechanisms present in vertebrates.
