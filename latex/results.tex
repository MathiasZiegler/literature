%%% TeX-master: "manuscript"
% !TeX spellcheck=en_GB

\section{Results}

\subsection{Paradoxical evolutionary correlation between NAD-dependent signalling and precursor metabolism}

% \to%do[inline]{MZ: Ich finde, die NAD-abhängigen Signalwege kommen in den Resultaten zu kurz weg. Es genügt vielleicht, einfach die 10 Klassen noch mal aufzuzählen (also z.B. x verschiedene PARP-Klassen, y Sirtuin-Klassen usw.). Dies würde ich gegenüberstellen an den Stellen, wo große Sprünge passieren. Daraus kann man z.B. ablesen, an welchen Stellen sich welche Signalwege speziell weiterentwickelt/diversifiziert haben. MB: Ich versuch mal, die entsprechenden Infos aus den Ergebnissen zu bekommen.}

To understand the functional roles and potential interplay between the three known enzymes that use Nam as substrate (NamPRT, NADA and NNMT), we first conducted a comprehensive phylogenetic analysis of these three enzymes. The phylogenetic distribution of the two enzymes that initiate the two different NAD salvage pathways, NADA and NamPRT is scattered in bacteria \citep{Gazzaniga2009}, while their co\-/occurrence has been detected in some marine invertebrates \citep{Gossmann2012FEBS}. As shown in Figure~\ref{fig:phylo_distribution}A, bacteria, fungi, and plants predominantly possess NADA and only a few of them harbour NamPRT. In contrast, Metazoa predominantly lost NADA and have NamPRT together with NNMT. NNMT seems to have arisen \textit{de~novo} or diverged rapidly in the most recent common ancestor of Ecdysozoa and Lophotrochozoa (fig.~\ref{fig:phylo_distribution}B). We were unable to find any \textit{NNMT} gene with an e-value below 0.1 in fungi or plants.

Nematodes are the only organisms, where we observed a concomitant presence of NADA and NNMT. In deuterostomes, the only large clade that possesses only NamPRT and seems to have lost NNMT are Sauropsida, and among them especially birds. The reason why about half of the sequenced bird genomes do not seem to encode for NNMT remains unclear. The distribution of NNMT in birds is quite scattered (suppl. fig.~S2) but could be explained by the fact that numerous bird genes are high in GC content \citep{Hron2015}. The lack of NNMT might be related to the differences in the excretion system, as the product of NNMT, methyl-Nam, is in mammals excreted with the urine. There are few metazoan species for which we could not find NamPRT or NADA, while NNMT was present. We assume that this is due to incomplete genomes in the database, as the distribution of such species is scarce and widely scattered.

In addition to the phylogenetic distribution of the two Nam salvage enzymes NADA and NamPRT, we analysed the phylogenetic diversity of enzymes catalysing NAD-dependent signalling reactions. To do so, we used the previously established classification into ten different families of NAD-consuming signalling enzymes \citep{Gossmann2012FEBS}. The detailed list of templates used for the phylogenetic analyses can be found in supplementary table~S1. The numbers shown in figure~\ref{fig:phylo_distribution}B denote the average number of NAD-dependent signalling enzyme families found in each clade. With the exception of Cnidaria and Lophotrochozoa, we find an average of three to four families in protostomes, whereas most deuterostome species have, on average, more than eight families with an increasing diversification of enzymes within some of these families \citep{Gossmann2014DNAR}.

Taken together, we found that NADA is lost in vertebrates, but strongly preserved in most other organisms, despite the higher energetic requirement of that pathway. Moreover, the selection for having both NamPRT and NNMT coincides with an increased diversification of NAD-dependent signalling. This observation seems counterintuitive, as one would expect that increased NAD-dependent signalling should be accompanied by an increase of substrate availability for NAD biosynthesis.


\subsection{Functional properties of NamPRT and NNMT have evolved to maximize NAD-dependent signalling}

To resolve this apparent contradiction, we wished to scrutinize the NAD metabolic network. Given the complexity of this network, we turned to modelling approaches and built a dynamic model of NAD metabolism based on previously reported kinetic data (for details, see experimental procedures and suppl. tab.~S2).

To be able to compare metabolic features of evolutionary quite different systems in our simulations and as we had limited information about species-specific expression levels of enzymes, we initially assumed equal expression rates for all enzymes. As we have very few cross species kinetic data, we were furthermore mainly relying on kinetic constants found for human or yeast enzymes. Wherever possible, we included both substrate affinities and known product inhibitions or inhibition by downstream metabolites. As we in addition assumed that cell growth is, besides NAD-consuming reactions, a major driving force for NAD biosynthesis, we analysed different growth rates (cell division rates) by simulating different dilution rates for all metabolites.

First, we addressed the unexpected correlation between the selection for co\-/occurrence of NamPRT and NNMT and an increase in the number of NAD-consuming enzyme families. We calculated steady state NAD concentrations and NAD consumption rates by simulating NAD biosynthesis proceeding via NamPRT in the presence or absence of NNMT. To achieve free NAD concentrations in the range reported in the literature and due to the very low turnover of NamPRT, we used tenfold higher NamPRT levels compared to other enzymes. We also adjusted the amount of NMNAT accordingly to avoid that the NAD synthesis rates are limited by this enzyme. Surprisingly, as shown in figure~\ref{fig:NNMT_NAD_flux}, the presence of NNMT enables higher rather than lower NAD consumption rates (fig.~\ref{fig:NNMT_NAD_flux}A). However, it diminishes the steady state concentration of NAD (fig.~\ref{fig:NNMT_NAD_flux}B). The decline in NAD concentration can be compensated by a higher expression of NamPRT, further increasing NAD consumption flux (dashed lines in fig.~\ref{fig:NNMT_NAD_flux}A and~B).

These results can be explained by looking in more detail at the kinetic parameters of NamPRT and NAD-consuming enzymes such as Sirtuin~1. Most NAD-consuming enzymes are inhibited by their product Nam. Thus, the presence of NNMT enables higher NAD consumption fluxes, by removing excess Nam from the cells. At the same time, the high substrate affinity of NamPRT maintains a sufficiently high NAD concentration, although the concentration is, as expected, lower than in the system without NNMT.

Kinetic parameters of NamPRT were previously measured for the human enzyme \citep{Burgos2008} as well as for some bacterial enzymes \citep{Sorci2010}, the latter having a much lower substrate affinity for Nam. We thus analysed the potential effect of NamPRT affinity ($K_{M}$) on NAD steady state concentration and NAD consumption flux. In the absence of NNMT, a variation of the substrate affinity of NamPRT for Nam has very little effect on steady state NAD concentration and NAD consumption flux (fig.~\ref{fig:NamPRT_affinity_Nam}A and~B). In the presence of NNMT, however, NAD consumption flux and NAD concentration increases with decreasing $K_{M}$ values of NamPRT (fig.~\ref{fig:NamPRT_affinity_Nam}C and~D).

Remarkably, NAD concentration and consumption flux are both considerably affected by cell division rates in a system without NNMT, at least if the enzyme expression is kept constant at different cell division rates. Of course, this is an artificial scenario, as one would assume organisms to regulate enzyme expression to achieve similar levels of metabolite concentrations instead. Nevertheless, in the absence of NNMT there seems to be a trade-off between maintainable NAD concentration and consumption flux. In contrast, in the presence of NNMT, NAD consumption rates and concentrations are almost independent of cell division rates.

Figures~\ref{fig:NamPRT_affinity_Nam}E and~F visualise a direct comparison of simulations assuming different affinities of NamPRT for Nam, in the presence or absence of NNMT. Interestingly, at an affinity of $K_{M}$ =~1\,$\mu$M, which is in the range of the $K_{M}$ of NADA for Nam, NAD consumption flux is only higher with NNMT present when cell division rates are low (fig.~\ref{fig:NamPRT_affinity_Nam}E). If the affinity of NamPRT is high enough, consumption rates are always higher with NNMT than without it. The NAD concentration is always lower with NNMT (fig.~\ref{fig:NamPRT_affinity_Nam}F).

To understand the interplay and competition for Nam between NNMT and NamPRT, we scanned a wide range of possible $K_{M}$ values for both enzymes in our simulations. As shown in figure~\ref{fig:optimal_substr_affinities}, the simulations indicate that both NAD consumption flux and NAD concentration would be minimal in case of a high $K_{M}$ for NamPRT and a low $K_{M}$ for NNMT. Conversely, lowering the $K_{M}$ of NamPRT to the nanomolar range substantially increases NAD consumption and concentration, which reach a maximum when the $K_{M}$ of NNMT is concomitantly elevated to the submillimolar range. The asterisks in figure~\ref{fig:optimal_substr_affinities} denote the $K_{M}$ values actually found for the human enzymes. Astonishingly, the naturally occurring $K_{M}$ values are very close to the theoretical optimum.


\subsection{Sequence variance acquired in metazoans enhances substrate affinity}

Given that NNMT might have exerted an evolutionary pressure on the development of NamPRT, one would expect to observe adaptations that are reflected in the NamPRT protein sequence arising shortly after the occurrence of NNMT. To explore this, we created a multiple sequence alignment. The alignment of selected sequences is shown in figure~\ref{fig:unresolved_loop}A and a more comprehensive multiple sequence alignment containing a larger number of species can be found in supplementary figure~S1. We found that most Deuterostomes that possess only NamPRT and NNMT (indicated by the blue circle) have an insert of ten amino acids corresponding to positions 42 to 51 in the human enzyme. This insert overlaps with a predicted weak nuclear localisation signal (NLS) that is lost when the insert is removed. These ten amino acids correspond to a stretch at the protein surface that is unresolved in all available crystal structures of human NamPRT (e.g. structure visualisation in fig.~\ref{fig:unresolved_loop}B from \citet{Wang2006}). Intriguingly, this presumed loop, depicted in red in figure~\ref{fig:unresolved_loop}B, is connected to one of the $\beta$-sheets involved in substrate binding \citep{Burgos2009} and in the functional homodimer, the two loops are placed side-by-side.

From these observations, we derived two possible hypotheses regarding the role of the loop in NamPRT function. The first hypothesis was that the presence of the loop could change the subcellular localisation of NamPRT, as it is overlapping with a predicted NLS. To test this hypothesis, we created a mutant NamPRT lacking the loop and recombinantly expressed FLAG-tagged wildtype and mutant NamPRT in HeLa S3 cells. Immunofluorescence imaging showed a mixed cytosolic nuclear localisation for both the wildtype and the mutant NamPRT (fig.~\ref{fig:unresolved_loop}C). Thus, deletion of the loop did not compromise the nuclear localisation.

The second hypothesis was that the sequence insertion might influence substrate binding of NamPRT. We thus expressed both wildtype and mutant proteins, N\=/terminally fused to a 6xHis-tag in \textit{E.~coli}, and purified them. The size exclusion chromatography profile showed that both wildtype and mutant protein were expressed as dimers (see suppl. fig.~S3), indicating that the missing residues in the mutant did not dramatically affect the protein structure. The enzymatic activity was measured by NMR spectroscopy using the detection of NMN produced in the presence and absence of ATP. Upon incubation with the NamPRT inhibitor FK866 \citep{Hasmann2003} for 30 minutes, both wildtype and mutant NamPRT did not produce any NMN, suggesting that binding of FK866 is not affected by the mutation (see suppl. fig.~S4). Nevertheless, the enzymatic activity of the mutant enzyme was only approximately 30\% of that of the wildtype enzyme (fig.~\ref{fig:unresolved_loop}D). In contrast to wildtype NamPRT, the activity of the mutant enzyme was not stimulated in the presence of ATP (fig.~\ref{fig:unresolved_loop}E). These observations suggest that the mutant enzyme is catalytically active, retains its dimeric state and sensitivity to FK866. However, the lower activity and the absence of catalytic activation by ATP indicate that deletion of amino acids 42 to 51 might have affected substrate binding.


\subsection{NamPRT and NNMT made NADA obsolete in vertebrates}

Next, we wished to understand why NADA was lost in vertebrates. As selection during evolution results usually from competition for resources, we built a two-compartment model, based on the pathway model described above. One compartment contains NADA, while the other one contains either NamPRT alone or together with NNMT. Both compartments share a limited Nam source (for details, see experimental procedures and suppl. tab.~S2). Without NNMT, the compartment containing NADA shows slightly lower NAD consumption rates (fig.~\ref{fig:NNMT_comp_advantage}A), but is able to maintain much higher NAD concentrations especially at low cell division rates (fig.~\ref{fig:NNMT_comp_advantage}B). At high cell division rates, steady state concentrations in both compartments are similar. This might explain why in bacteria, that often have relatively high growth rates, both systems can co-exist.

In the presence of NNMT, the NamPRT compartment has both higher NAD consumption rates and higher steady state NAD concentrations than the compartment containing NADA (fig.~\ref{fig:NNMT_comp_advantage}C and~D). The higher NAD concentrations in the compartment containing NamPRT and NNMT can, however, only be maintained if the affinity of NamPRT for Nam is high enough. If the substrate affinity of NamPRT is too low (high $K_{M}$), the NADA compartment is able to maintain higher NAD concentrations, but still has a lower NAD consumption flux. Taken together, the results suggest that the NADA pathway might have become obsolete upon emergence of the high affinity NamPRT. This in turn might have been induced by the appearance of NNMT.
