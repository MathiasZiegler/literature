%%% TeX-master: "manuscript"
% !TeX spellcheck=en_GB

\noindent
{\huge\sffamily\bfseries NamPRT and NNMT – evolutionary and kinetic drivers of NAD-dependent signalling \par}

\vspace{5mm}

\noindent
Mathias Bockwoldt\textsuperscript{1}, Dorothée Houry\textsuperscript{2}, Marc Niere\textsuperscript{2}, Toni I. Gossmann\textsuperscript{3}, Mathias Ziegler\textsuperscript{2}, and Ines Heiland\textsuperscript{1,§}

\vspace{1cm}

\noindent
\textsuperscript{1}Department of Arctic and Marine Biology, UiT The Arctic University of Norway, Biologibygget, Framstredet 39, 9017 Tromsø, Norway

\noindent
\textsuperscript{2}Department of Biomedicine, University of Bergen, Jonas Lies Vei 91, 5020 Bergen, Norway

\noindent
\textsuperscript{3}Department of Animal and Plant Sciences, Western Bank, University of Sheffield, Sheffield, S10 2TN, United Kingdom

\noindent
§ Corresponding author: ines.heiland@uit.no


\section*{Summary}

NAD is best known as cofactor in redox reactions, but it is also substrate of NAD-dependent signalling reactions that consume NAD and release nicotinamide (Nam). Two different Nam salvage pathways exist. We conducted extensive phylogenetic analyses of these pathways and show that in lower organisms the initial deamidation of Nam is prevalent, whilst the direct conversion of Nam to the mononucleotide by Nam phosphoribosyltransferase (NamPRT) dominates in animals and eventually remains as the single Nam recycling route in vertebrates. Strikingly, loss of the deamidation pathway in early vertebrates is preceded by the emergence of a new enzyme that marks Nam for excretion by methylation – nicotinamide N-methyltransferase (NNMT). Paradoxically, the occurrence of this Nam degrading enzyme is paralleled by a diversification of NAD dependent signalling enzymes in vertebrates.

To better understand these evolutionary changes and the role of NNMT for NAD metabolism, we built a mathematical model of the pathway using available enzyme kinetics data. Our simulations indicate that NNMT is required to maintain high NAD consumption fluxes, thereby enabling diversification of the NAD-dependent signalling pathways. This kinetic regulation requires and explains the unusually high substrate affinity of the key enzyme for Nam salvage, NamPRT. Moreover, we suggest that NNMT exerted evolutionary pressure on NamPRT, enforcing the development of its high substrate affinity. Using multiple sequence alignments, we identified a sequence insertion, first occurring in vertebrates, that parallels an experimentally verified increase in the substrate affinity of the enzyme. Additional simulations show that the deamidation pathway became obsolete owing to the high substrate affinity of NamPRT. Collectively, our results illustrate a close evolutionary relationship between NAD biosynthesis and the diversification of NAD-dependent signalling pathways, potentially driven by the concomitant occurrence of a regulator of Nam salvage, NNMT.


%\section*{Keywords}
