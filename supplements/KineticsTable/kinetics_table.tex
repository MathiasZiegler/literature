\documentclass[a4paper,10pt]{article}
\usepackage[utf8]{inputenc}
\usepackage{amsmath}
\usepackage{graphicx}
\usepackage{epsfig}
\usepackage{booktabs,colortbl,tabularx}
\usepackage{longtable}
\usepackage{float}
\usepackage{footmisc}
\usepackage{breqn}
\restylefloat{table}


\usepackage[pdftex,bookmarks=false,
pdftex,
%linkcolor=black,
linkcolor=blue,
citecolor=blue,
colorlinks=true,                %%% hyper-references for pdflatex
urlcolor=blue]{hyperref}

\begin{document}

\section*{Supplementary Table 2}
\setcounter{table}{1}
\begin{longtable}{p{1.5cm}p{1.5cm}p{2.5cm}cc}
\multicolumn{5}{l}{\textbf{Overview of kinetic constants used for the
construction of the model.}} \cr\toprule
  \cr\textbf{Enzyme} &
  \textbf{EC\linebreak number}&
  \textbf{Kinetic\linebreak parameter} &
  \textbf{References} &
  \textbf{Rate Law} \\
\cr\hline
\endfirsthead
\multicolumn{5}{c}{Table 1 -- Continued}
\cr\hline
\cr\textbf{Enzyme} &
 \textbf{EC\linebreak number}&
  \textbf{Kinetic\linebreak parameter} &
  \textbf{Reference} &
  \textbf{Rate Law} \\
\cr\hline
\endhead
NADA & \href{http://www.chem.qmul.ac.uk/iubmb/enzyme/EC3/5/1/19.html}{3.5.1.19} &$K_M$:9.6$\mu$M  & \cite{Smith2012} & Product inhibition\\
& & $K_{iP}$:120$\mu$M  & &
    \\
    & & $k_{cat}$:0.65$s^{-1}$  & &
    \\
    \hline
NADS & \href{http://www.chem.qmul.ac.uk/iubmb/enzyme/EC6/3/5/1.html}{6.3.5.1}
&$K_M$:190$\mu$M  & \cite{Yi1972} & HMM\\
    & & $k_{cat}$:21$s^{-1}$  & &
    \\  \hline
NMNAT &
\href{http://www.chem.qmul.ac.uk/iubmb/enzyme/EC2/7/7/1.html}{2.7.7.1}  
&$K_{M_{NaMN}}$:67.7$\mu$M  & \cite{Sorci2007}\footnote{Values for NMNAT1 used}
& Substrate Competition\\
&  \href{http://www.chem.qmul.ac.uk/iubmb/enzyme/EC2/7/7/18.html}{2.7.7.18}   & $k_{{cat}_{NaMN}}$:42.9$s^{-1}$  & & \\
    & & $K_{M_{NMN}}$:22.3$\mu$M & & \\
    & & $k_{{cat}_{NMN}}$:53.8$s^{-1}$  & & \\
	& & $K_{M_{NMN}}$:59$\mu$M  & & \\
    & & $k_{{cat}_{NAD}}$:129.1$s^{-1}$  &\cite{Berger2005}\footnote{Keq used
    for calculation of turnover rate of reverse reaction} &
    \\
    & & $K_{M_{NaAD}}$:502$\mu$M  & & \\
    & & $k_{{cat}_{NaAD}}$:103.8$s^{-1}$  &\cite{Berger2005}\footnote{Equilibrium constant used for calculation of turnover rate of reverse reaction} &
    \\ \hline
NMNT & \href{http://www.chem.qmul.ac.uk/iubmb/enzyme/EC2/1/1/1.html}{2.1.1.1}
&$K_M$:400$\mu$M  & \cite{Aksoy1994} & Product inhibition\\
& & $K_{iP}$:60$\mu$M  & &
    \\
    & & $k_{cat}$:8.1$s^{-1}$  & \cite{Alston1988}&
    \\  \hline
NamPT & \href{http://www.chem.qmul.ac.uk/iubmb/enzyme/EC6/3/5/1.html}{6.3.5.1}
&$K_M$:5nM  & \cite{Burgos2008} & Competitive inhibition\\
    & & $k_{cat}$:0.0077$s^{-1}$  & &
    \\
     & & $K_{i_{NAD}}$:  2.1$\mu$M& &
    \\  \hline
    NAPRT &
       \href{http://www.chem.qmul.ac.uk/iubmb/enzyme/EC2/4/2/11.html}{2.4.2.11}
    &$K_M$:1.5$\mu$M  & \cite{Burgos2008} & HMM\\
    & & $k_{cat}$:3.3$s^{-1}$  & &
    \\  \hline
    SIRT1 &
    \href{http://www.chem.qmul.ac.uk/iubmb/enzyme/EC3/5/1/index.html}{3.5.1.-}
    &$K_M$:29$\mu$M  & \cite{Borra2004} & Product inhibition\\
    & & $K_{iP}$:60$\mu$M  & &
    \\
    & & $k_{cat}$:0.67$s^{-1}$  & &
    \\  \hline
    NT5 &
    \href{http://www.chem.qmul.ac.uk/iubmb/enzyme/EC3/1/3/5.html}{3.1.3.5}
    &$K_{M_{NaMN}}$:3.5mM  & \cite{Kulikova2015} & HMM \\
    & & $k_{{cat}_{NaMN}}$:2.8$s^{-1}$
    & & \\
    & &$K_{M_{NMN}}$: 5mM   &  \\
     & & $k_{{cat}_{NMN}}$:0.5$s^{-1}$
     & & \\  \hline
     PNP &
    \href{http://www.chem.qmul.ac.uk/iubmb/enzyme/EC3/2/4/2.html}{2.4.2.1}
    &$K_M$:1.48mM  & \cite{Wielgus-Kutrowska1997} & HMM \\
    & & $k_{cat}$:40$s^{-1}$
    & & \\  \hline
    NRK &
    \href{http://www.chem.qmul.ac.uk/iubmb/enzyme/EC2/7/1/173.html}{2.7.1.173}
    &$K_M$:3.4$\mu$M & \cite{Dolle2009} & HMM \\
    & & $k_{cat}$:0.23$s^{-1}$
    & & \\  \hline
  \bottomrule
  \label{tab:kinetic}

\end{longtable}

\subsection*{Amount of enzymes and import rates}
The total enzyme concentration was set to 10 times the scaling factor, for all enzymes except NamPT and NMNAT, for which the concentration was set to 100 times the scaling factor. As enzyme concentrations here have an arbitrary unit a scaling factor of 0.1$\mu$M  was applied to all enzymatic reactions to achieve consumption rates that are in the range of reported values  \cite{Liu2018}. Concentration of potential co-substrates were assumed to be constant and not-limiting for the reaction. Thus being implicitly represented by maximal velocities consisting of total enzyme concentration times tunrover rates. Nam import rates for import into the system was set to 0.1 $\mu$M/s for all simulations, being in the range meassured for Nam uptake in mammalian cells \cite{Namuptake}. In addition to the reactions listed above an additional NAD consumption was simulated using HMM-kinetics with a susbtrate affinity of 0.3 mM and a turnover rate of 1. Furthermore, reversible NAD binding to proteins was simulated using reversible mass actions kinetics with an equilibrium constant of 0.1, which is in a range of values reported in the literature, dissociation and association constants where set to 10 and 100$s^{-1}$ respectively.
s
For the two compartment simulation, compartmentsize was equal for both compartments and set to 1$\mu$l. The actual compartmentsize does not change the outcome of the simulations as long as both compartments have equal volumes. The Nam import rates were set to 100$s^{-1}$ for both compartments. The amount of NADA present was set to 100. Thus equal to the amount of NamPT used. 

\subsection*{Kinetic Rate Laws}


\subsubsection*{Product Inhibition}
\begin{equation}
v=\cfrac{E_T\cdot k_{cat}\cdot S}{K_M + S + \cfrac{K_M\cdot P}{K_{iP}} }
\end{equation}

\subsubsection*{Competitive Inhibition}
\begin{equation}
v=\cfrac{E_T\cdot k_{cat}\cdot S}{K_M + S + \cfrac{K_M\cdot I}{K_{iI}} }
\end{equation}


\subsubsection*{Henry-Michaelis Menten for irreversible reactions (HMM)}
\begin{equation}
v=\frac{E_T\cdot k_{cat}\cdot S}{K_M + S}
\end{equation}


\subsubsection*{Substrate Competition at NMNAT}
\begin{equation}
v=E_T \cdot \cfrac{\cfrac{k_{{cat}_{A}}\cdot A \cdot
B}{K_{{M}_A}}-\cfrac{k_{{cat}_{P}}\cdot P \cdot
Q}{K_{{M}_P}}}{1+\cfrac{A}{K_{{M}_A}}+\cfrac{B}{K_{{M}_B}}+\cfrac{P}{K_{{M}_P}}+\cfrac{Q}{K_{{M}_Q}}}
\end{equation}



\bibliographystyle{plos2009}
\bibliography{tablelibrary}
\end{document}
