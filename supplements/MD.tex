\documentclass[a4paper,10pt]{article}
\usepackage[utf8]{inputenc}
\usepackage[left=3cm]{geometry}
\usepackage{booktabs}
\usepackage{textcomp}
\usepackage{graphicx}
\begin{document}
\setcounter{table}{2}
\begin{table}
\section*{Supplementary Table 3}
 \begin{tabular}{l|cccccc}
  \toprule
  %\hline
  PDB Code & 2H3D & 3DGR & 3DHD & 3DHF & 3DKJ & 3DKL \\ 
  \midrule
  %\hline
  2H3D & - & 0.95 & 0.85 & 0.86 & 0.88 & 0.88 \\
  %\hline
  3DGR &   &  -   & 0.61 & 0.61 & 0.55 & 0.57 \\
  %\hlineM
  3DHD &   &      &  -   & 0.43 & 0.40 & 0.43 \\
  %\hline
  3DHF &   &      &      &  -   & 0.42 & 0.33 \\
  %\hline
  3DKJ &   &      &      &      &  -   & 0.39 \\
  %\hline
  3DKL &   &      &      &      &      &  -   \\
  %\hline
 \bottomrule
 \end{tabular}
  \caption{Root mean square deviation (RMSD) values between different structures (in \AA). The alignment and RMSD calculation was done with PyMOL\cite{PyMOL}. The structures are 2H3D (human NAMPT) \cite{Wang2006}, 3DGR (human NAMPT$\cdot$AMPcP complex) \cite{Burgos2009}, 3DHD (human NAMPT$\cdot$NMN$\cdot$Mg$_2$PPi complex) \cite{Burgos2009}, 3DHF (human BeF$_{3^-}$-NAMPT$\cdot$NMN$\cdot$Mg$_2$PPi complex) \cite{Burgos2009}, 3DKJ (human NAMPT$\cdot$PRPP$\cdot$BzAM complex) \cite{Burgos2009}, and 3DKL (human BeF$_{3^-}$-NAMPT$\cdot$Mg$_2$PRPP$\cdot$BzAM complex) \cite{Burgos2009}. The structural resolution of the PDB structures ranges from 1.8\,\AA~to 2.1\,\AA.}
\end{table}




\bibliographystyle{plos2009}
\bibliography{MD}

\end{document}
