\documentclass[a4paper,10pt]{article} \usepackage[utf8x]{inputenc}
\usepackage{amsmath}
\usepackage{graphicx}
\usepackage{epsfig}
\usepackage{booktabs,colortbl,tabularx}
\usepackage{longtable}
\usepackage{float}
\usepackage{footmisc}
\usepackage{breqn}
\restylefloat{table}


\usepackage[pdftex,bookmarks=false,
pdftex,
%linkcolor=black,
linkcolor=blue,
citecolor=blue,
colorlinks=true,                %%% hyper-references for pdflatex
urlcolor=blue]{hyperref}

\begin{document}

\section*{Supplementary Table 1}
\addcontentsline{toc}{section}{Supplementary Table 1}
\begin{longtable}{p{3cm}p{2cm}p{3cm}cc}
\multicolumn{5}{l}{\textbf{Overview of kinetic constants used for the
construction of the model.}} \cr\toprule
  \cr\textbf{Enzyme} &
  \textbf{EC\linebreak number}&
  \textbf{Kinetic\linebreak parameter} &
  \textbf{References} &
  \textbf{Rate Law} \\
\cr\hline
\endfirsthead
\multicolumn{5}{c}{Table 1 -- Continued}
\cr\hline
\cr\textbf{Enzyme} &
 \textbf{EC\linebreak number}&
  \textbf{Kinetic\linebreak parameter} &
  \textbf{Reference} &
  \textbf{Rate Law} \\
\cr\hline
\endhead
NADA & \href{http://www.chem.qmul.ac.uk/iubmb/enzyme/EC3/5/1/19.html}{3.5.1.19} &$K_M$:9.6$\mu$M  & \cite{Smith2012} & Product inhibition\\
& & $K_{iP}$:120$\mu$M  & & 
    \\
    & & $k_{cat}$:0.65$s^{-1}$  & & 
    \\
    \hline
NADS & \href{http://www.chem.qmul.ac.uk/iubmb/enzyme/EC6/3/5/1.html}{6.3.5.1}
&$K_M$:190$\mu$M  & \cite{Yi1972} & HMM\\
    & & $k_{cat}$:21$s^{-1}$  & & 
    \\  \hline
NMNAT &
\href{http://www.chem.qmul.ac.uk/iubmb/enzyme/EC2/7/7/1.html}{2.7.7.1}/\href{http://www.chem.qmul.ac.uk/iubmb/enzyme/EC2/7/7/18.html}{2.7.7.18}
&$K_{M_{NaMN}}$:67.7$\mu$M  & \cite{Sorci2007}\footnote{Values for NMNAT1 used}
& Substrate Competition\\
    & & $k_{{cat}_{NaMN}}$:42.9$s^{-1}$  & & \\
    & & $K_{M_{NMN}}$:22.3$\mu$M & & \\
    & & $k_{{cat}_{NMN}}$:53.8$s^{-1}$  & & \\
	& & $K_{M_{NMN}}$:59$\mu$M  & & \\
    & & $k_{{cat}_{NAD}}$:129.1$s^{-1}$  &\cite{Berger2005}\footnote{Keq used
    for calculation of turnover rate of reverse reaction} &
    \\
    & & $K_{M_{NaAD}}$:502$\mu$M  & & \\
    & & $k_{{cat}_{NaAD}}$:103.8$s^{-1}$  &\cite{Berger2005}\footnote{Equilibrium constant used for calculation of turnover rate of reverse reaction} &
    \\ \hline
NMNT & \href{http://www.chem.qmul.ac.uk/iubmb/enzyme/EC2/1/1/1.html}{2.1.1.1}
&$K_M$:400$\mu$M  & \cite{Aksoy1994} & Product inhibition\\
& & $K_{iP}$:60$\mu$M  & & 
    \\
    & & $k_{cat}$:8.1$s^{-1}$  & \cite{Alston1988}& 
    \\  \hline
NamPRT & \href{http://www.chem.qmul.ac.uk/iubmb/enzyme/EC6/3/5/1.html}{6.3.5.1}
&$K_M$:5nM  & \cite{Burgos2008} & HMM\\
    & & $k_{cat}$:0.0077$s^{-1}$  & & 
    \\  \hline
    NAPRT &
       \href{http://www.chem.qmul.ac.uk/iubmb/enzyme/EC2/4/2/11.html}{2.4.2.11}
    &$K_M$:23$\mu$M  & \cite{Hanna1983} & HMM\\
    & & $k_{cat}$:3.3$s^{-1}$  & & 
    \\  \hline
    SIRT1 &
    \href{http://www.chem.qmul.ac.uk/iubmb/enzyme/EC3/5/1/index.html}{3.5.1.-}
    &$K_M$:29$\mu$M  & \cite{Borra2004} & Product inhibition\\
    & & $K_{iP}$:60$\mu$M  & & 
    \\
    & & $k_{cat}$:0.67$s^{-1}$  & & 
    \\  \hline
    NT5 &
    \href{http://www.chem.qmul.ac.uk/iubmb/enzyme/EC3/1/3/5.html}{3.1.3.5}
    &$K_M$:100$\mu$M  & \cite{} & HMM \\
    & & $k_{cat}$:1$s^{-1}$ 
    & & \\  \hline   
  \bottomrule
  \label{tab:kinetic}
\end{longtable}

The total enzyme concentration was set to 10 for all enzymes if not mentioned
otherwise. Concentration of potentiantial cosubstrate was assumed to be constant
and considered to be not limitng the reaction and thus represented by the
maximal velocities given.

\subsection*{Kinetic Rate Laws}


\subsubsection*{Product Inhibition}
\begin{equation}
v=\cfrac{E_T\cdot k_{cat}\cdot S}{K_M + S + \cfrac{K_M\cdot P}{K_{iP}} }
\end{equation}


\subsubsection*{Henry-Michaelis Menten for irreversible reactions (HMM)}
\begin{equation}
v=\frac{E_T\cdot k_{cat}\cdot S}{K_M + S}
\end{equation}


\subsubsection*{Substrate Competition at NMNAT}
\begin{equation}
v=E_T \cdot \cfrac{\cfrac{k_{{cat}_{A}}\cdot A \cdot
B}{K_{{M}_A}}-\cfrac{k_{{cat}_{P}}\cdot P \cdot
Q}{K_{{M}_P}}}{1+\cfrac{A}{K_{{M}_A}}+\cfrac{B}{K_{{M}_B}}+\cfrac{P}{K_{{M}_P}}+\cfrac{Q}{K_{{M}_Q}}}
\end{equation}



\bibliographystyle{plos2009}
\bibliography{TableS1}
\end{document}